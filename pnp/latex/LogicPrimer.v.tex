\begin{coqdoccode}
\coqdocemptyline
\coqdocemptyline
\end{coqdoccode}

\chapter{Propositional Logic}
\label{ch:logic}
 



In the previous chapter we had an opportunity to explore Coq as a
functional programming language and learn how to define inductive
datatypes and programs that operate with them, implementing the latter
ones directly or using the automatically-generated recursion
combinators. Importantly, most of the values that we met until this
moment, inhabited the types, which were defined as elements of the
sort \coqdockw{Set}. The types \coqdocvar{unit}, \coqdocvar{empty}, \coqdocvar{nat}, \coqdocvar{nat} \ensuremath{\times} \coqdocvar{unit} etc. are
good examples of \textit{first-order} types inhabiting the sort \coqdockw{Set} and,
therefore, contributing to the analogy between sets and first-order
types, which we explored previously.  In this chapter, we will be
working with a new kind of entities, incorporated by Coq:
\textit{propositions}.


\section{Propositions and the \texorpdfstring{\protect\coqdockw{Prop}}{Prop} sort}


\label{sec:propsort}


In Coq, propositions bear a lot of similarities with types,
demonstrated in Chapter~\ref{ch:funprog}, and inhabit a separate
sort \coqdockw{Prop}, similarly to how first-order types inhabit
\coqdockw{Set}.\footnote{In the Coq community, the datatypes of  \coqdockw{Prop} 
sort are usually referred to as \emph{inductive
predicates}.\index{inductive predicates}} The ``values'' that have
elements of \coqdockw{Prop} as their types are usually referred to as \textit{proofs}
or \textit{proof terms}, the naming convention which stems out of the idea of
\index{Curry-Howard correspondence} \emph{Curry-Howard
Correspondence}~\cite{Curry:34,Howard:80}.\footnote{\url{http://en.wikipedia.org/wiki/Curry-Howard_correspondence}}
Sometimes, the Curry-Howard Correspondence is paraphrased as
\textit{proofs-as-programs}, which is truly illuminating when it comes to the
intuition behind the formal proof construction in Coq, which, in fact,
is just programming in disguise.


\index{Calculus of Inductive Constructions}
\index{CIC|see {Calculus~of~Inductive~Constructions}}
\index{intuitionistic type theory}
\index{Martin-\loef's type theory|see {intuitionistic type theory}}
\index{Prop sort@\texttt{Prop} sort}


The \textit{Calculus of Inductive Constructions}
(CIC)~\cite{Bertot-Casteran:BOOK,Coquand-Huet:IC88} a logical
foundation of Coq, similarly to its close relative, Martin-\loef's
\textit{Intuitionistic Type Theory} \cite{Martin-Loef:84}, considers proofs
to be just regular values of the ``programming'' language it
defines. Therefore, the process of constructing proofs in Coq is
very similar to the process of writing programs. Intuitively, when
one asks a question ``Whether the proposition \coqdocaxiom{P} is \textit{true}?'', what is
meant in fact is ``Whether the \textit{proof} of \coqdocaxiom{P} can be
constructed?''. This is an unusual twist, which is crucial for
understanding the concept of the ``truth'' and proving propositions in
CIC (and, equivalently, in Coq), so we specifically outline it here in
the form of a motto:



\begin{center}
Only those propositions are considered to be \emph{true}, which are
provable \emph{constructively},\\ i.e., by providing an \emph{explicit} proof term,
that inhabits them.
\end{center}



This formulation of ``truth'' is somewhat surprising at the first
encounter, comparing to classical propositional logic, where the
propositions are considered to be true simply if they are tautologies
(i.e., reduce to the boolean value \coqdocvar{true} for all possible
combinations of their free variables' values), therefore leading to
the common proof method in classical propositional logic: truth
tables.  While the truth table methodology immediately delivers the
recipe to prove propositions without quantifiers \textit{automatically} (that
is, just by checking the corresponding truth tables), it does not
quite scale when it comes to the higher-order propositions (i.e.,
quantifying over predicates) as well as of propositions quantifying
over elements of arbitrary domains. For instance, the following
proposition, in which the reader can recognize the induction principle
over natural numbers, cannot be formulated in the zeroth- or
first-order propositional logic (and, in fact, in \textit{any} propositional
logic):



\begin{center}
For any predicate $P$, if $P(0)$ holds, and for any $m$, $P(m)$
implies $P(m + 1)$, \\
then for any $n$, $P(n)$ holds.
\end{center}



The statement above is \textit{second-order} as it binds a first-order
predicate by means of universal quantification, which makes it belong
to the corresponding second-order logic (which is not even
propositional, as it quantifies over arbitrary natural values, not
just propositions). Higher-order logics~\cite{Church:JSL40} are
known to be undecidable in general, and, therefore, there is no
automatic way to reduce an arbitrary second-order formula to one of
the two values: \coqdocvar{true} or \coqdocvar{false}.


CIC as a logic is expressive enough to accommodate propositions with
quantifications of an arbitrary order and over arbitrary values. On
one hand, it makes it an extremely powerful tool to state almost any
proposition of interest in modern mathematics or computer science. On
the other hand, proving such statements (i.e., constructing their
proof terms), will require human assistance, in the same way the
``paper-and-pencil'' proofs are constructed in classical
mathematics. However, unlike the paper-and-pencil proofs, proofs
constructed in Coq are a subject of immediate \textit{automated} check, since
they are just programs to be verified for well-typedness. Therefore,
the process of proof construction in Coq is \textit{interactive} and assumes
the constant interoperation between a human prover, who constructs a
proof term for a proposition (i.e., writes a program), and Coq, the
proof assistant, which carries out the task of \textit{verifying} the proof
(i.e., type-checking the program). This largely defines our agenda for
the rest of this course: we are going to see how to \textit{prove} logical
statements by means of writing \textit{programs}, that have the types
corresponding to these statements.


In the rest of this chapter we will focus only on the capability of
Coq as a formal system allowing one to reason about propositions,
leaving reasoning about values aside till the next chapter. It is
worth noticing that a fragment of Coq, which deals with the sort
\coqdockw{Prop}, accommodating all the propositions, and allows the programmer
to make statements with propositions, corresponds to the logical
calculus, known as System~$F_{\omega}$\index{System $F_{\omega}$}
(see Chapter 30 of~\cite{Pierce:BOOK02}) \index{System $F$}
extending System $F$~\cite{Reynolds:SP74,Girard:PhD}, mentioned in
Chapter~\ref{ch:funprog}. Unlike System $F$, which introduces
polymorphic types, and, equivalently, first-order propositions that
quantify over other propositions, System~$F_{\omega}$ allows one to
quantify as well over \textit{type operators}, which can be also thought of
as higher-order propositions.


\section{The truth and the falsehood in Coq}




We start our acquaintance with propositional logic in Coq by
demonstrating how the two simplest propositions, the truth and the
falsehood, are encoded. Once again, let us remember that, unlike in
propositional logic, in Coq these two are \textit{not} the only possible
propositional \textit{values}, and soon we will see how a wide range of
propositions different from mere truth or falsehood are
implemented. From now on, we will be always including to the
development the standard Ssreflect's module \coqdoclibrary{ssreflect},
\ssrm{ssreflect} which imports some necessary machinery for dealing
with propositions and proofs.


 \begin{coqdoccode}
\coqdocemptyline
\coqdocnoindent
\coqdockw{\textsf{{From}}} \coqdocvar{mathcomp} \coqdockw{Require} \coqdockw{Import} \coqdocvar{ssreflect}.\coqdoceol
\coqdocemptyline
\coqdocemptyline
\end{coqdoccode}


The truth is represented in Coq as a datatype of sort \coqdockw{Prop} with just
one constructor, taking no arguments:


\begin{coqdoccode}
\coqdocemptyline
\coqdocnoindent
\coqdockw{Print} \coqdocvar{True}.\coqdoceol
\coqdocemptyline
\end{coqdoccode}
\coqdoceol
\coqdocemptyline
\coqdocnoindent
\coqdockw{Inductive} \coqdocinductive{True} : \coqdockw{Prop} :=  \coqdocconstructor{I} : \coqdocinductive{True}

\coqdocemptyline


\ssrd{True}


Such simplicity makes it trivial to construct an instance of the
\coqdocinductive{True} proposition:\footnote{In the context of propositional logic, we
will be using the words ``type'' and ``proposition'' interchangeably
without additional specification when it's clear from the context.}
Now we can prove the following proposition in Coq's embedded
propositional logic, essentially meaning that \coqdocinductive{True} is provable.


\begin{coqdoccode}
\coqdocemptyline
\coqdocnoindent
\coqdockw{Theorem} \coqdocvar{true\_is\_true}: \coqdocvar{True}.\coqdoceol
\coqdocemptyline
\end{coqdoccode}


\coqdoceol
\coqdocemptyline
\coqdocnoindent
1 \coqdockw{subgoals}, \coqdockw{subgoal} 1 (\coqdocvar{ID} 1)\coqdoceol
\coqdocindent{1.00em}
\coqdoceol
\coqdocindent{1.00em}
============================\coqdoceol
\coqdocindent{1.50em}
\coqdocinductive{True}

\coqdocemptyline


The command \coqdockw{Theorem} \ccom{Theorem} serves two purposes. First, similarly to the
command \coqdockw{Definition}, it defines a named entity, which is not
necessarily a proposition. In this case the name is
\coqdoclemma{true\_is\_true}. Next, similarly to \coqdockw{Definition}, there might follow a
list of parameters, which is empty in this example. Finally, after the
colon : there is a type of the defined value, which in this case it
\coqdocinductive{True}. With this respect there is no difference between \coqdockw{Theorem} and
\coqdockw{Definition}. However, unlike \coqdockw{Definition}, \coqdockw{Theorem} doesn't require
one to provide the expression of the corresponding type right
away. Instead, the \textit{interactive proof mode} \index{interactive proof mode}  
is activated, so the proof term could be constructed
incrementally. The process of the gradual proof construction is what
makes Coq to be a \textit{interactive proof assistant}, in addition to being
already a programming language with dependent types.


Although not necessary, it is considered a good programming practice
in Coq to start any interactive proof with the Coq's command
\ccom{Proof} \coqdockw{Proof}, which makes the final scripts easier to read
and improves the general proof layout.


\begin{coqdoccode}
\coqdocemptyline
\coqdocnoindent
\coqdockw{Proof}.\coqdoceol
\coqdocemptyline
\end{coqdoccode}


\index{goal} In the interactive proof mode, the \coqdocvar{goals} display
shows a \textit{goal} of the proof---the type of the value to be constructed
(\coqdocinductive{True} in this case), which is located below the double line. Above
the line one can usually see the context of \textit{assumptions}, which can
be used in the \index{assumption} process of constructing the
proof. Currently, the assumption context is empty, as the theorem we
stated does not make any and ventures to prove \coqdocinductive{True} out of thin
air. Fortunately, this is quite easy to do, as from the formulation of
the \coqdocinductive{True} type we already know that it is inhabited by its only
constructor \coqdocconstructor{I}. The next line proved the \textit{exact} value of the type of
the goal.\ttac{exact:}


\begin{coqdoccode}
\coqdocemptyline
\coqdocnoindent
\coqdoctac{exact}: \coqdocvar{I}.\coqdoceol
\coqdocemptyline
\end{coqdoccode}


This completes the proof, as indicated by the Proof General's
\texttt{*response*} display:


\coqdoceol
\coqdocemptyline
\coqdocnoindent
\coqdocvar{No} \coqdocvar{more} \coqdockw{subgoals}.\coqdoceol
\coqdocnoindent
(\coqdoctac{dependent} \coqdocvar{evars}:)

\coqdocemptyline


The only thing left to complete the proof is to inform Coq that now
the theorem \coqdoclemma{true\_is\_true} is proved, which is achieved by typing the
command \ccom{Qed} \coqdockw{Qed}.


\begin{coqdoccode}
\coqdocemptyline
\coqdocnoindent
\coqdockw{Qed}.\coqdoceol
\coqdocemptyline
\end{coqdoccode}


In fact, typing \coqdockw{Qed} invokes a series of additional checks, which
ensure the well-formedness of the constructed proof term. Although the
proof of \coqdoclemma{true\_is\_true} is obviously valid, in general, there is a
number of proof term properties to be checked \textit{a posteriori} and
particularly essential in the case of proofs about infinite objects,
which we do not cover in these course (see Chapter~13
of~\cite{Bertot-Casteran:BOOK} for a detailed discussion on such
proofs).


So, our first theorem is proved. As it was hinted previously, it could
have been stated even more concisely, formulated as a mere definition,
and proved by means of providing a corresponding value, without the
need to enter the proof mode:


\begin{coqdoccode}
\coqdocemptyline
\coqdocnoindent
\coqdockw{Definition} \coqdocvar{true\_is\_true'}: \coqdocvar{True} := \coqdocvar{I}.\coqdoceol
\coqdocemptyline
\end{coqdoccode}


Although this is a valid way to prove statements in Coq, it is not as
convenient as the interactive proof mode, when it comes to
construction of large proofs, arising from complicated
statements. This is why, when it comes to proving propositions, we
will prefer the interactive proof mode to the ``vanilla'' program
definition. It is worth noticing, though, that even though the process
of proof construction in Coq usually looks more like writing a
\textit{script}, consisting from a number of commands (which are called
\textit{tactics} in Coq jargon),\index{Coq/Ssreflect
tactics}\index{tactics} \index{tactics|seealso {Coq/Ssreflect
tactics}} the result of such script, given that it eliminates all of
the goals, is a valid well-typed Coq program. In comparison, in some
other dependently-typed frameworks (e.g., in Agda\index{Agda}), the
construction of proof terms does not obscure the fact that what is
being constructed is a program, so the resulting interactive proof
process is formulated as ``filling the holes'' in a program (i.e., a
proof-term), which is being gradually refined. We step away from the
discussion on which of these two views to the proof term construction
is more appropriate.


There is one more important difference between values defined as
\coqdockw{Definition}s \ccom{Definition}\ccom{Theorem} and \coqdockw{Theorem}s. While
both define what in fact is a proof terms for the declared type, the
value bound by \coqdockw{Definition} is \textit{transparent}: it can be executed by
means of unfolding and subsequent evaluation of its body. In contrast,
a proof term bound by means of \coqdockw{Theorem} is \textit{opaque}, which means that
its body cannot be evaluated and serves only one purpose: establish
the fact that the corresponding type (the theorem's statement) is
inhabited, and, therefore is true.  This distinction between
definitions and theorems arises from the notion of \textit{proof
irrelevance}, which, informally, states that (ideally) one shouldn't
be able to distinguish between two proofs of the same statement as
long as they both are valid.\footnote{Although, in fact, proof terms
in Coq can be very well distinguished.} Conversely, the programs
(that is, what is created using the \coqdockw{Definition} command) are
typically of interest by themselves, not only because of the type they
return.


The difference between the two definitions of the truth's validity,
which we have just constructed, can be demonstrated by means of the
\coqdockw{Eval} command.


\begin{coqdoccode}
\coqdocemptyline
\coqdocnoindent
\coqdockw{Eval} \coqdoctac{compute} \coqdoctac{in} \coqdocvar{true\_is\_true}.\coqdoceol
\end{coqdoccode}
 = \coqdoclemma{true\_is\_true} : \coqdocinductive{True}
\begin{coqdoccode}
\coqdocemptyline
\coqdocnoindent
\coqdockw{Eval} \coqdoctac{compute} \coqdoctac{in} \coqdocvar{true\_is\_true'}.\coqdoceol
\end{coqdoccode}
 = \coqdocconstructor{I} : \coqdocinductive{True}


As we can see now, the theorem is evaluated to itself, whereas the
definition evaluates to it body, i.e., the value of the constructor
\coqdocconstructor{I}.  






A more practical analogy for the above distinction can be drawn if one
will think of \coqdockw{Definition}s as of mere functions, packaged into
libraries and intended to be used by third-party clients. In the same
spirit, one can think of \coqdockw{Theorem}s as of facts that need to be
checked only once when established, so no one would bother to re-prove
them again, knowing that they are valid, and just appeal to their
types (statement) without exploring the proof.\footnote{While we
consider this to be a valid analogy to the mathematical community
functions, it is only true in spirit. In the real life, the statements
proved once, are usually re-proved by students for didactical reasons,
in order to understand the proof principles and be able to produce
other proofs. Furthermore, the history of mathematics witnessed a
number of proofs that have been later invalidated. Luckily, the
mechanically-checked proofs are usually not a subject of this
problem.} This is similar to what is happening during the oral
examinations on mathematical disciplines: a student is supposed to
remember the statements of theorems from the \textit{previous} courses and
semesters, but is not expected to reproduce their proofs.


At this point, an attentive reader can notice that the definition of
\coqdocinductive{True} in Coq is strikingly similar to the definition of the type
\coqdocvar{unit} from Chapter~\ref{ch:funprog}. This is a fair observation,
which brings us again to the Curry-Howard analogy, and makes it
possible to claim that the trivial truth proposition is isomorphic to
the \coqdocvar{unit} type from functional programming. Indeed, both have just
one way to be constructed and can be constructed in any context, as
their single constructor does not require any arguments.


Thinking by analogy, one can now guess how the falsehood can be encoded.


\ssrd{False}
\begin{coqdoccode}
\coqdocemptyline
\coqdocnoindent
\coqdockw{Print} \coqdocvar{False}.\coqdoceol
\coqdocemptyline
\end{coqdoccode}
\coqdoceol
\coqdocemptyline
\coqdocnoindent
\coqdockw{Inductive} \coqdocinductive{False} : \coqdockw{Prop} :=  

\coqdocemptyline


Unsurprisingly, the proposition \coqdocinductive{False} in Coq is just a Curry-Howard
counterpart of the type \coqdocvar{empty}, which we have constructed in
Chapter~\ref{ch:funprog}. Moreover, the same intuition that was
applicable to \coqdocvar{empty}'s recursion principle (``anything can be produced
given an element of an empty set''), is applicable to reasoning by
induction with the \coqdocinductive{False} proposition:


\begin{coqdoccode}
\coqdocemptyline
\coqdocnoindent
\coqdockw{Check} \coqdocvar{False\_ind}.\coqdoceol
\coqdocemptyline
\end{coqdoccode}
\coqdoceol
\coqdocemptyline
\coqdocnoindent
\coqdocdefinition{False\_ind}\coqdoceol
\coqdocindent{2.50em}
: \coqdockw{\ensuremath{\forall}} \coqdocaxiom{P} : \coqdockw{Prop}, \coqdocinductive{False} \ensuremath{\rightarrow} \coqdocaxiom{P}

\coqdocemptyline


That is, \textit{any} proposition can be derived from the falsehood by means
of implication.\footnote{In light of the Curry-Howard analogy, at
\index{Curry-Howard correspondence} this moment it shouldn't come as a
surprise that Coq uses the arrow notation \texttt{->} both for
function types and for propositional implication: after all, they both
are just particular cases of functional abstraction, in sorts
\texttt{Set} or \texttt{Prop}, correspondingly.} For instance, we can
prove now that \coqdocinductive{False} implies the equality 1 = 2.\footnote{We
postpone the definition of the equality till the next chapter, and for
now let us consider it to be just an arbitrary proposition.}


\begin{coqdoccode}
\coqdocemptyline
\coqdocnoindent
\coqdockw{Theorem} \coqdocvar{one\_eq\_two}: \coqdocvar{False} \ensuremath{\rightarrow} 1 = 2.\coqdoceol
\coqdocnoindent
\coqdockw{Proof}.\coqdoceol
\coqdocemptyline
\end{coqdoccode}


One way to prove this statement is to use the \coqdocinductive{False} induction
principle, i.e., the theorem \coqdocdefinition{False\_ind}, directly by instantiating it
with the right predicate \coqdocaxiom{P}:


\begin{coqdoccode}
\coqdocemptyline
\coqdocnoindent
\coqdoctac{exact}: (\coqdocvar{False\_ind} (1 = 2)).\coqdoceol
\coqdocemptyline
\end{coqdoccode}


This indeed proves the theorem, but for now, let us explore a couple
of other ways to prove the same statement. For this we first
\ccom{Undo} \coqdockw{Undo} the last command of the already succeeded but not
yet completed proof.


\begin{coqdoccode}
\coqdocemptyline
\coqdocnoindent
\coqdockw{Undo}.\coqdoceol
\coqdocemptyline
\end{coqdoccode}


Instead of supplying the argument (1 = 2) to \coqdocdefinition{False\_ind} manually,
we can leave it to Coq to figure out, what it should be, by using the
Ssreflect \coqdoctac{apply}: tactic.\ssrt{apply:}


\begin{coqdoccode}
\coqdocemptyline
\coqdocnoindent
\coqdoctac{apply}: \coqdocvar{False\_ind}.\coqdoceol
\coqdocemptyline
\end{coqdoccode}


The following thing just happened: the tactic \coqdoctac{apply}: supplied with
an argument \coqdocdefinition{False\_ind}, tried to figure out whether our goal \coqdocinductive{False}
\ensuremath{\rightarrow} (1 = 2) matches any \textit{head} type of the theorem \coqdocdefinition{False\_ind}.
\index{head type} By \textit{head type} we mean a component of type (in
this case, \coqdockw{\ensuremath{\forall}} \coqdocaxiom{P} : \coqdockw{Prop}, \coqdocinductive{False} \ensuremath{\rightarrow} \coqdocaxiom{P}), which is a type by itself
and possibly contains free variables. For instance, recalling that
\ensuremath{\rightarrow} is right-associative, head-types of \coqdocdefinition{False\_ind} would be \coqdocaxiom{P},
\coqdocinductive{False} \ensuremath{\rightarrow} \coqdocaxiom{P} and \coqdockw{\ensuremath{\forall}} \coqdocaxiom{P} : \coqdockw{Prop}, \coqdocinductive{False} \ensuremath{\rightarrow} \coqdocaxiom{P} itself.


So, in our example, the call to the tactics \coqdoctac{apply}: \coqdocdefinition{False\_ind} makes
Coq realize that the goal we are trying to prove matches the type
\coqdocinductive{False} \ensuremath{\rightarrow} \coqdocaxiom{P}, where \coqdocaxiom{P} is taken to be (1 = 2). Since in this case
there is no restrictions on what \coqdocaxiom{P} can be (as it is
universally-quantified in the type of \coqdocdefinition{False\_ind}), Coq assigns \coqdocaxiom{P} to
be (1 = 2), which, after such specialization, turns the type of
\coqdocdefinition{False\_ind} to be exactly the goal we're after, and the proof is done.


There are many more ways to prove this rather trivial statement, but
at this moment we will demonstrate just yet another one, which does
not appeal to the \coqdocdefinition{False\_ind} induction principle, but instead
proceeds by \textit{case analysis}.


\begin{coqdoccode}
\coqdocemptyline
\coqdocnoindent
\coqdockw{Undo}.\coqdoceol
\coqdocemptyline
\coqdocnoindent
\coqdoctac{case}.\coqdoceol
\coqdocemptyline
\end{coqdoccode}


The tactic \coqdoctac{case}\ssrt{case} makes Coq to perform the case
analysis. In particular, it \textit{deconstructs} the \textit{top assumption} of the
goal. The top assumption in the goal is such that it comes first
before any arrows, and in this case it is a value of type
\coqdocinductive{False}. Then, for all constructors of the type, whose value is being
case-analysed, the tactic \coqdoctac{case} constructs \textit{subgoals} to be
proved. Informally, in mathematical reasoning, the invocation of the
\coqdoctac{case} tactic would correspond to the statement ``let us consider all
possible cases, which amount to the construction of the top
assumption''. Naturally, since \coqdocinductive{False} has \textit{no} constructors (as it
corresponds to the \coqdocvar{empty} type), the case analysis on it produces
\textit{zero} subgoals, which completes the proof immediately. Since the
result of the proof is just some program, again, we can demonstrate
the effect of \coqdoctac{case} tactic by proving the same theorem with an exact
proof term:


\begin{coqdoccode}
\coqdocemptyline
\coqdocnoindent
\coqdockw{Undo}.\coqdoceol
\coqdocemptyline
\coqdocnoindent
\coqdoctac{exact}: (\coqdockw{fun} (\coqdocvar{f}: \coqdocvar{False}) \ensuremath{\Rightarrow} \coqdockw{match} \coqdocvar{f} \coqdockw{with} \coqdockw{end}).\coqdoceol
\coqdocemptyline
\end{coqdoccode}


As we can see, one valid proof term of \coqdoclemma{one\_eq\_two} is just a
function, which case-analyses on the value of type \coqdocinductive{False}, and such
case-analysis has no branches.


\begin{coqdoccode}
\coqdocemptyline
\coqdocnoindent
\coqdockw{Qed}.\coqdoceol
\coqdocemptyline
\end{coqdoccode}
\section{Implication and universal quantification}




By this moment we have already seen how implication is represented in
Coq: it is just a functional type, represented by the ``arrow'' notation
\ensuremath{\rightarrow} and familiar to all functional programmers. Indeed, if a function
of type \coqdocvariable{A} \ensuremath{\rightarrow} \coqdocvar{B} is a program that takes an argument value of type \coqdocvariable{A}
and returns a result value of type \coqdocvar{B}, then the propositional
implication \coqdocaxiom{P} \ensuremath{\rightarrow} \coqdocaxiom{Q} is, ... a program that takes an argument proof
term of type \coqdocaxiom{P} and returns a proof of the proposition \coqdocaxiom{Q}.


Unlike most of the value-level functions we have seen so far,
propositions are usually parametrized by other propositions, which
makes them instances of \textit{polymorphic} types, as they appear in
\index{System $F$}
System~$F$ and System $F_{\omega}$. Similarly to these systems, in
Coq the universal quantifier \coqdockw{\ensuremath{\forall}} (spelled \texttt{forall}) binds a
variable immediately following it in the scope of the subsequent
type.\footnote{As it has been noticed in Chapter~\ref{ch:funprog} the
$\forall$-quantifier is Coq's syntactic notation for dependent
function types, sometimes also referred to a \emph{$\Pi$-types} or
\emph{dependent product types}.}\index{dependent function type} For
instance, the transitivity of implication in Coq can be expressed via
the following proposition:


\begin{center}
\coqdockw{\ensuremath{\forall}} \coqdocaxiom{P} \coqdocaxiom{Q} \coqdocdefinition{$R$}: \coqdockw{Prop}, (\coqdocaxiom{P} \ensuremath{\rightarrow} \coqdocaxiom{Q}) \ensuremath{\rightarrow} (\coqdocaxiom{Q} \ensuremath{\rightarrow} \coqdocdefinition{$R$}) \ensuremath{\rightarrow} \coqdocaxiom{P} \ensuremath{\rightarrow} \coqdocdefinition{$R$}
\end{center}


The proposition is therefore \textit{parametrized} over three propositional
variables, \coqdocaxiom{P}, \coqdocaxiom{Q} and \coqdocdefinition{$R$}, and states that from a proof term of
type \coqdocaxiom{P} \ensuremath{\rightarrow} \coqdocaxiom{Q} and a proof term of type \coqdocaxiom{Q} \ensuremath{\rightarrow} \coqdocdefinition{$R$} one can build a
proof term of type \coqdocaxiom{P} \ensuremath{\rightarrow} \coqdocdefinition{$R$}.\footnote{Recall that the arrows have
right associativity, just like function types in Haskell and OCaml,
which allows one to apply functions partially, specifying their
arguments one by one} Let us now prove this statement in the form of
a theorem.  \begin{coqdoccode}
\coqdocemptyline
\coqdocnoindent
\coqdockw{Theorem} \coqdocvar{imp\_trans}: \coqdockw{\ensuremath{\forall}} \coqdocvar{P} \coqdocvar{Q} \coqdocvar{$R$}: \coqdockw{Prop}, (\coqdocvar{P} \ensuremath{\rightarrow} \coqdocvar{Q}) \ensuremath{\rightarrow} (\coqdocvar{Q} \ensuremath{\rightarrow} \coqdocvar{$R$}) \ensuremath{\rightarrow} \coqdocvar{P} \ensuremath{\rightarrow} \coqdocvar{$R$}.\coqdoceol
\coqdocnoindent
\coqdockw{Proof}.\coqdoceol
\coqdocemptyline
\end{coqdoccode}


Our goal is the statement of the theorem, its type. The first thing we
are going to do is to ``peel off'' some of the goal assumptions---the
\coqdockw{\ensuremath{\forall}}-bound variables---and move them from the goal to the
assumption context (i.e., from below to above the double line). This
step in the proof script is usually referred to as \textit{bookkeeping},
since it does not directly contribute to reducing the goal, but
instead moves some of the values from the goal to assumption, as a
preparatory step for the future reasoning.


\index{tacticals}
\index{bookkeeping}
\index{tacticals|seealso {Coq/Ssreflect tacticals}}
Ssreflect offers a tactic and a small but powerful toolkit of
\textit{tacticals} (i.e., higher-order tactics) for bookkeeping. In
particular, for moving the bound variables from ``bottom to the top'',
one should use a combination of the ``no-op'' tactic \coqdoctac{move}\ssrt{move}
and the tactical \ensuremath{\Rightarrow} \ssrtl{=>}(spelled \texttt{=}>). The following
command moves the next three assumptions from the goal, \coqdocaxiom{P}, \coqdocaxiom{Q} and
\coqdocdefinition{$R$} to the assumption context, simultaneously renaming them to \coqdocvariable{A},
\coqdocvar{B} and \coqdocvar{C}. The renaming is optional, so we just show it here to
demonstrate the possibility to give arbitrary (and, preferably, more
meaningful) names to the assumption variables ``on the fly'' while
constructing the proof via a script.


\begin{coqdoccode}
\coqdocemptyline
\coqdocnoindent
\coqdoctac{move}\ensuremath{\Rightarrow} \coqdocvar{A} \coqdocvar{B} \coqdocvar{C}.\coqdoceol
\end{coqdoccode}
\coqdoceol
\coqdocemptyline
\coqdocindent{1.00em}
\coqdocvariable{A} : \coqdockw{Prop}\coqdoceol
\coqdocindent{1.00em}
\coqdocvar{B} : \coqdockw{Prop}\coqdoceol
\coqdocindent{1.00em}
\coqdocvar{C} : \coqdockw{Prop}\coqdoceol
\coqdocindent{1.00em}
============================\coqdoceol
\coqdocindent{1.50em}
(\coqdocvariable{A} \ensuremath{\rightarrow} \coqdocvar{B}) \ensuremath{\rightarrow} (\coqdocvar{B} \ensuremath{\rightarrow} \coqdocvar{C}) \ensuremath{\rightarrow} \coqdocvariable{A} \ensuremath{\rightarrow} \coqdocvar{C}

\coqdocemptyline


We can now move the three other arguments to the top using the same
command: the \coqdoctac{move}\ensuremath{\Rightarrow} combination works uniformly for \coqdockw{\ensuremath{\forall}}-bound
variables as well as for the propositions on the left of the arrow.


\begin{coqdoccode}
\coqdocemptyline
\coqdocnoindent
\coqdoctac{move}\ensuremath{\Rightarrow} \coqdocvar{H1} \coqdocvar{H2} \coqdocvar{a}.\coqdoceol
\end{coqdoccode}
\coqdoceol
\coqdocemptyline
\coqdocindent{1.00em}
\coqdocvar{H1} : \coqdocvariable{A} \ensuremath{\rightarrow} \coqdocvar{B}\coqdoceol
\coqdocindent{1.00em}
\coqdocvar{H2} : \coqdocvar{B} \ensuremath{\rightarrow} \coqdocvar{C}\coqdoceol
\coqdocindent{1.00em}
\coqdocvariable{a} : \coqdocvariable{A}\coqdoceol
\coqdocindent{1.00em}
============================\coqdoceol
\coqdocindent{1.50em}
\coqdocvar{C}

\coqdocemptyline


Again, there are multiple ways to proceed now. For example, we can
recall the functional programming and get the result of type \coqdocvar{C} just
by two subsequent applications of \coqdocvar{H1} and \coqdocvar{H2} to the value \coqdocvariable{a} of type \coqdocvariable{A}:


\begin{coqdoccode}
\coqdocemptyline
\coqdocnoindent
\coqdoctac{exact}: (\coqdocvar{H2} (\coqdocvar{H1} \coqdocvar{a})).\coqdoceol
\coqdocemptyline
\end{coqdoccode}


Alternatively, we can replace the direct application of the hypotheses
\coqdocvar{H1} and \coqdocvar{H2} by the reversed sequence of calls to the \coqdoctac{apply}:
tactics.


\begin{coqdoccode}
\coqdocemptyline
\coqdocnoindent
\coqdockw{Undo}.\coqdoceol
\coqdocemptyline
\end{coqdoccode}
\noindent
The first use of \coqdoctac{apply}: will replace the goal \coqdocvar{C} by the goal \coqdocvar{B},
since this is what it takes to get \coqdocvar{C} by using \coqdocvar{H2}:


\begin{coqdoccode}
\coqdocemptyline
\coqdocnoindent
\coqdoctac{apply}: \coqdocvar{H2}.\coqdoceol
\end{coqdoccode}


\coqdoceol
\coqdocemptyline
\coqdocindent{1.00em}
\coqdocvar{H1} : \coqdocvariable{A} \ensuremath{\rightarrow} \coqdocvar{B}\coqdoceol
\coqdocindent{1.00em}
\coqdocvariable{a} : \coqdocvariable{A}\coqdoceol
\coqdocindent{1.00em}
============================\coqdoceol
\coqdocindent{1.50em}
\coqdocvar{B}

\coqdocemptyline
\noindent
The second use of \coqdoctac{apply}: reduces the proof of \coqdocvar{B} to the proof of
\coqdocvariable{A}, demanding an appropriate argument for \coqdocvar{H1}.


\begin{coqdoccode}
\coqdocemptyline
\coqdocnoindent
\coqdoctac{apply}: \coqdocvar{H1}.\coqdoceol
\end{coqdoccode}
\coqdoceol
\coqdocemptyline
\coqdocindent{1.00em}
\coqdocvariable{a} : \coqdocvariable{A}\coqdoceol
\coqdocindent{1.00em}
============================\coqdoceol
\coqdocindent{1.50em}
\coqdocvariable{A}

\coqdocemptyline


Notice that both calls to \coqdoctac{apply}: removed the appropriate hypotheses,
\coqdocvar{H1} and \coqdocvar{H2} from the assumption context. If one needs a hypothesis
to stay in the context (to use it twice, for example), then the
occurrence of the tactic argument hypothesis should be parenthesised:
\coqdoctac{apply}: (\coqdocvar{H1}).


Finally, we can see that the only goal left to prove is to provide a
proof term of type \coqdocvariable{A}. Luckily, this is exactly what we have in the
assumption by the name \coqdocvariable{a}, so the following demonstration of the
exact \coqdocvariable{a} finishes the proof:


\begin{coqdoccode}
\coqdocemptyline
\coqdocnoindent
\coqdoctac{exact}: \coqdocvar{a}.\coqdoceol
\coqdocnoindent
\coqdockw{Qed}.\coqdoceol
\coqdocemptyline
\end{coqdoccode}


In the future, we will replace the use of trivial tactics, such as
\coqdoctac{exact}: by Ssreflect's much more powerful tactics \coqdocvar{\texttt{\emph{done}}},\ssrt{done} which
combines a number of standard Coq's tactics in an attempt to finish
the proof of the current goal and reports an error if it fails to do
so. 



\begin{exercise}[$\forall$-distributivity]
\label{ex:forall-dist}

Formulate and prove the following theorem in Coq, which states the
distributivity of universal quantification with respect to implication:
\[
\forall P~Q, [(\forall x, P(x) \implies Q(x)) \implies ((\forall y, P(y)) \implies \forall z, Q(z))]
\]

\hint Be careful with the scoping of universally-quantified variables
and use parentheses to resolve ambiguities!

\end{exercise}

\begin{coqdoccode}
\coqdocemptyline
\coqdocemptyline
\end{coqdoccode}


\subsection{On forward and backward reasoning}




Let us check now the actual value of the proof term of theorem
\coqdoclemma{imp\_trans}. 


\begin{coqdoccode}
\coqdocemptyline
\coqdocnoindent
\coqdockw{Print} \coqdocvar{imp\_trans}.\coqdoceol
\end{coqdoccode}


\coqdoceol
\coqdocemptyline
\coqdocnoindent
\coqdoclemma{imp\_trans} = \coqdoceol
\coqdocindent{1.00em}
\coqdockw{fun} (\coqdocvariable{A} \coqdocvar{B} \coqdocvar{C} : \coqdockw{Prop}) (\coqdocvar{H1} : \coqdocvariable{A} \ensuremath{\rightarrow} \coqdocvar{B}) (\coqdocvar{H2} : \coqdocvar{B} \ensuremath{\rightarrow} \coqdocvar{C}) (\coqdocvariable{a} : \coqdocvariable{A}) \ensuremath{\Rightarrow}\coqdoceol
\coqdocindent{1.00em}
(\coqdockw{fun} \coqdocvar{\_evar\_0\_} : \coqdocvar{B} \ensuremath{\Rightarrow} \coqdocvar{H2} \coqdocvar{\_evar\_0\_}) ((\coqdockw{fun} \coqdocvar{\_evar\_0\_} : \coqdocvariable{A} \ensuremath{\Rightarrow} \coqdocvar{H1} \coqdocvar{\_evar\_0\_}) \coqdocvariable{a})\coqdoceol
\coqdocindent{3.50em}
: \coqdockw{\ensuremath{\forall}} \coqdocaxiom{P} \coqdocaxiom{Q} \coqdocdefinition{$R$} : \coqdockw{Prop}, (\coqdocaxiom{P} \ensuremath{\rightarrow} \coqdocaxiom{Q}) \ensuremath{\rightarrow} (\coqdocaxiom{Q} \ensuremath{\rightarrow} \coqdocdefinition{$R$}) \ensuremath{\rightarrow} \coqdocaxiom{P} \ensuremath{\rightarrow} \coqdocdefinition{$R$}\coqdoceol
\coqdocnoindent
\coqdoceol
\coqdocnoindent
\coqdocvar{Argument} \coqdocvar{scopes} \coqdocvar{are} [\coqdocvar{type\_scope} \coqdocvar{type\_scope} \coqdocvar{type\_scope} \coqdocvar{\_} \coqdocvar{\_} \coqdocvar{\_}]

\coqdocemptyline


Even though the proof term looks somewhat hairy, this is almost
exactly our initial proof term from the first proof attempt: \coqdocvar{H2} (\coqdocvar{H1}
\coqdocvariable{a}). The only difference is that the hypotheses \coqdocvar{H1} and \coqdocvar{H2} are
\textit{eta-expanded},\index{eta-expansion} that is instead of simply \coqdocvar{H1}
the proof terms features its operational equivalent \coqdockw{fun} \coqdocvariable{b}: \coqdocvar{B} \ensuremath{\Rightarrow} \coqdocvar{H2}
\coqdocvariable{b}. Otherwise, the printed program term indicates that the proof
obtained by means of direct application of \coqdocvar{H1} and \coqdocvar{H2} is the same
(modulo eta-expansion) as the proof obtained by means of using the
\coqdoctac{apply}: tactic.


These two styles of proving: by providing a direct proof to the goal
or some part of it, and by first reducing the goal via tactics, are
usually referred in the mechanized proof community as \textit{forward} and
\textit{backward} proof styles\index{forward proof style}\index{backward
proof style}.



\begin{itemize}
\item  The \textit{backward} proof style assumes that the goal is being gradually
  transformed by means of applying some tactics, until its proof
  becomes trivial and can be completed by means of basic tactics,
  like \coqdoctac{exact}: or \coqdocvar{\texttt{\emph{done}}}.



\item  The \textit{forward} proof style assumes that the human prover has some
  ``foresight'' with respect to the goal she is going to prove, so she
  can define some ``helper'' entities as well as to adapt the available
  assumptions, which will then be used to solve the goal. Typical
  example of the forward proofs are the proofs from the classical
  mathematic textbooks: first a number of ``supporting'' lemmas is
  formulated, proving some partial results, and finally all these
  lemmas are applied in concert in order to prove an important
  theorem.

\end{itemize}


While the standard Coq is very well supplied with a large number of
tactics that support reasoning in the backward style, it is less
convenient for the forward-style reasoning. This aspect of the tool is
significantly enhanced by Ssreflect, which introduces a small number
of helping tactics, drastically simplifying the forward proofs, as we
will see in the subsequent chapters.


\subsection{Refining and bookkeeping assumptions}




Suppose, we have the following theorem to prove, which is just a
simple reformulation of the previously proved \coqdoclemma{imp\_trans}:
\begin{coqdoccode}
\coqdocemptyline
\coqdocnoindent
\coqdockw{Theorem} \coqdocvar{imp\_trans'} (\coqdocvar{P} \coqdocvar{Q} \coqdocvar{$R$}: \coqdockw{Prop}) : (\coqdocvar{Q} \ensuremath{\rightarrow} \coqdocvar{$R$}) \ensuremath{\rightarrow} (\coqdocvar{P} \ensuremath{\rightarrow} \coqdocvar{Q}) \ensuremath{\rightarrow} \coqdocvar{P} \ensuremath{\rightarrow} \coqdocvar{$R$}.\coqdoceol
\coqdocnoindent
\coqdockw{Proof}.\coqdoceol
\coqdocnoindent
\coqdoctac{move}\ensuremath{\Rightarrow} \coqdocvar{H1} \coqdocvar{H2}.\coqdoceol
\coqdocemptyline
\end{coqdoccode}


Notice that we made the propositional variables \coqdocaxiom{P}, \coqdocaxiom{Q} and \coqdocdefinition{$R$} to be
parameters of the theorem, rather than \coqdockw{\ensuremath{\forall}}-quantified
values. This relieved us from the necessity to lift them using
\coqdoctac{move}\ensuremath{\Rightarrow} in the beginning of the proof.


In is natural to expect that the original \coqdoclemma{imp\_trans} will be of some
use. We are now in the position to apply it directly, as the current
goal matches its conclusion. However, let us do something slightly
different: \textit{move} the statement of \coqdoclemma{imp\_trans} into the goal,
simultaneously with specifying it (or, equivalently, partially
applying) to the assumptions \coqdocvar{H1} and \coqdocvar{H2}. Such move ``to the bottom
part'' in Ssreflect is implemented by means of the \ssrtl{:} :
tactical, following the \coqdoctac{move} command:


\begin{coqdoccode}
\coqdocemptyline
\coqdocnoindent
\coqdoctac{move}: (\coqdocvar{imp\_trans} \coqdocvar{P} \coqdocvar{Q} \coqdocvar{$R$})=> \coqdocvar{H}.\coqdoceol
\end{coqdoccode}


\coqdoceol
\coqdocemptyline
\coqdocindent{1.00em}
\coqdocvar{H1} : \coqdocaxiom{Q} \ensuremath{\rightarrow} \coqdocdefinition{$R$}\coqdoceol
\coqdocindent{1.00em}
\coqdocvar{H2} : \coqdocaxiom{P} \ensuremath{\rightarrow} \coqdocaxiom{Q}\coqdoceol
\coqdocindent{1.00em}
\coqdocvar{H} : (\coqdocaxiom{P} \ensuremath{\rightarrow} \coqdocaxiom{Q}) \ensuremath{\rightarrow} (\coqdocaxiom{Q} \ensuremath{\rightarrow} \coqdocdefinition{$R$}) \ensuremath{\rightarrow} \coqdocaxiom{P} \ensuremath{\rightarrow} \coqdocdefinition{$R$}\coqdoceol
\coqdocindent{1.00em}
============================\coqdoceol
\coqdocindent{1.50em}
\coqdocaxiom{P} \ensuremath{\rightarrow} \coqdocdefinition{$R$}

\coqdocemptyline


What has happened now is a good example of the forward reasoning: the
specialized version of (\coqdoclemma{imp\_trans} \coqdocaxiom{P} \coqdocaxiom{Q} \coqdocdefinition{$R$}), namely, (\coqdocaxiom{P} \ensuremath{\rightarrow} \coqdocaxiom{Q}) \ensuremath{\rightarrow} (\coqdocaxiom{Q} \ensuremath{\rightarrow}
\coqdocdefinition{$R$}) \ensuremath{\rightarrow} \coqdocaxiom{P} \ensuremath{\rightarrow} \coqdocdefinition{$R$}, has been moved to the goal, so it became ((\coqdocaxiom{P} \ensuremath{\rightarrow} \coqdocaxiom{Q}) \ensuremath{\rightarrow}
(\coqdocaxiom{Q} \ensuremath{\rightarrow} \coqdocdefinition{$R$}) \ensuremath{\rightarrow} \coqdocaxiom{P} \ensuremath{\rightarrow} \coqdocdefinition{$R$}) \ensuremath{\rightarrow} \coqdocaxiom{P} \ensuremath{\rightarrow} \coqdocdefinition{$R$}. Immediately after that, the top
assumption (that is, what has been just ``pushed'' to the goal stack)
was moved to the top and given the name \coqdocvar{H}. Now we have the
assumption \coqdocvar{H} that can be applied in order to reduce the goal.  


\begin{coqdoccode}
\coqdocemptyline
\coqdocnoindent
\coqdoctac{apply}: \coqdocvar{H}.\coqdoceol
\end{coqdoccode}


\coqdoceol
\coqdocemptyline
\coqdocindent{1.00em}
\coqdocvar{H1} : \coqdocaxiom{Q} \ensuremath{\rightarrow} \coqdocdefinition{$R$}\coqdoceol
\coqdocindent{1.00em}
\coqdocvar{H2} : \coqdocaxiom{P} \ensuremath{\rightarrow} \coqdocaxiom{Q}\coqdoceol
\coqdocindent{1.00em}
============================\coqdoceol
\coqdocindent{1.50em}
\coqdocaxiom{P} \ensuremath{\rightarrow} \coqdocaxiom{Q}\coqdoceol
\coqdocnoindent
\coqdoceol
\coqdocnoindent
\coqdockw{subgoal} 2 (\coqdocvar{ID} 142) \coqdockw{is}:\coqdoceol
\coqdocindent{0.50em}
\coqdocaxiom{Q} \ensuremath{\rightarrow} \coqdocdefinition{$R$}

\coqdocemptyline


The proof forked into two goals, since \coqdocvar{H} had two arguments, which we
can now fulfill separately, as they trivially are our assumptions.  


\begin{coqdoccode}
\coqdocemptyline
\coqdocnoindent
\coqdocvar{\texttt{\emph{done}}}.\coqdoceol
\coqdocnoindent
\coqdocvar{\texttt{\emph{done}}}.\coqdoceol
\coqdocemptyline
\end{coqdoccode}


The proof is complete, although the last step is somewhat repetitive,
since we know that for two generated sub-goals the proofs are the
same. In fact, applications of tactics can be \textit{chained} using the ;
\ssrtl{;}connective, so the following complete proof of \coqdoclemma{imp\_trans'}
runs \coqdocvar{\texttt{\emph{done}}} for \textit{all} subgoals generated by \coqdoctac{apply}:
\coqdocvar{H}\ccom{Restart}:


\begin{coqdoccode}
\coqdocemptyline
\coqdocnoindent
\coqdockw{Restart}.\coqdoceol
\coqdocemptyline
\coqdocnoindent
\coqdoctac{move}: (\coqdocvar{imp\_trans} \coqdocvar{P} \coqdocvar{Q} \coqdocvar{$R$})=> \coqdocvar{H} \coqdocvar{H1} \coqdocvar{H2}.\coqdoceol
\coqdocnoindent
\coqdoctac{apply}: \coqdocvar{H}; \coqdocvar{\texttt{\emph{done}}}.\coqdoceol
\coqdocemptyline
\end{coqdoccode}


Also, notice that the sequence in which the hypotheses were moved to
the top has changed: in order to make the proof more concise, we first
created the ``massaged'' version of \coqdoclemma{imp\_trans}, and then moved it as
\coqdocvar{H} to the top, following by \coqdocvar{H1} and \coqdocvar{H2}, which were in the goal
from the very beginning.


To conclude this section, let us demonstrate even shorter way to prove
this theorem once again.


\begin{coqdoccode}
\coqdocemptyline
\coqdocnoindent
\coqdockw{Restart}.\coqdoceol
\coqdocnoindent
\coqdoctac{move}\ensuremath{\Rightarrow}\coqdocvar{H1} \coqdocvar{H2}; \coqdoctac{apply}: (\coqdocvar{imp\_trans} \coqdocvar{P} \coqdocvar{Q} \coqdocvar{$R$})=>//.\coqdoceol
\coqdocnoindent
\coqdockw{Qed}.\coqdoceol
\coqdocemptyline
\end{coqdoccode}


After traditional move of the two hypotheses to the top, we applied
the specialised version of \coqdoclemma{imp\_trans}, where its three first
arguments were explicitly instantiated with the local \coqdocaxiom{P}, \coqdocaxiom{Q} and
\coqdocdefinition{$R$}. This application generated two subgoals, each of which has been
then automatically solved by the trailing tactical \ssrtl{//}\ensuremath{\Rightarrow}
//, which is equivalent to ;\coqdoctac{try} \coqdocvar{\texttt{\emph{done}}} and, informally speaking,
``tries to kill all the newly created goals''.\footnote{The Coq's
\texttt{try}\ssrtl{try} tactical tries to execute its tactic argument in a "soft
way", that is, not reporting an error if the argument fails.}




\section{Conjunction and disjunction}


\label{sec:conjdisj}


Two main logical connectives, conjunction and disjunction, are
implemented in Coq as simple inductive predicates in the sort
\coqdockw{Prop}. In order to avoid some clutter, from this moment and till the
end of the chapter let us start a new module \coqdocvar{Connectives} and assume
a number of propositional variables in it (as we remember, those will
be abstracted over outside of the module in the statements
they\ccom{Variables} happen to occur).


\begin{coqdoccode}
\coqdocemptyline
\coqdocnoindent
\coqdockw{Module} \coqdocvar{Connectives}.\coqdoceol
\coqdocnoindent
\coqdockw{Variables} \coqdocvar{P} \coqdocvar{Q} \coqdocvar{$R$}: \coqdockw{Prop}.\coqdoceol
\coqdocemptyline
\end{coqdoccode}


The propositional conjunction of \coqdocaxiom{P} and \coqdocaxiom{Q}, denoted by \coqdocaxiom{P} \ensuremath{\land} \coqdocaxiom{Q}, is
a straightforward Curry-Howard counterpart of the \coqdocvar{pair} datatype that
we have already seen in Chapter~\ref{ch:funprog}, and is defined by
means of the predicate \coqdocinductive{and}.


\begin{coqdoccode}
\coqdocemptyline
\coqdocnoindent
\coqdockw{Locate} "\_ /\symbol{92} \_".\coqdoceol
\coqdocemptyline
\end{coqdoccode}
"\coqdocvariable{A} \ensuremath{\land} \coqdocvar{B}" := \coqdocinductive{and} \coqdocvariable{A} \coqdocvar{B}  : \coqdocvar{type\_scope} \begin{coqdoccode}
\coqdocemptyline
\coqdocnoindent
\coqdockw{Print} \coqdocvar{and}.\coqdoceol
\coqdocemptyline
\end{coqdoccode}


\ssrd{and}


\coqdoceol
\coqdocemptyline
\coqdocnoindent
\coqdockw{Inductive} \coqdocinductive{and} (\coqdocvariable{A} \coqdocvar{B} : \coqdockw{Prop}) : \coqdockw{Prop} :=  \coqdocconstructor{conj} : \coqdocvariable{A} \ensuremath{\rightarrow} \coqdocvar{B} \ensuremath{\rightarrow} \coqdocvariable{A} \ensuremath{\land} \coqdocvar{B}\coqdoceol
\coqdocnoindent
\coqdoceol
\coqdocnoindent
\coqdocvar{For} \coqdocconstructor{conj}: \coqdocvar{Arguments} \coqdocvariable{A}, \coqdocvar{B} \coqdocvar{are} \coqdocvar{implicit}\coqdoceol
\coqdocnoindent
\coqdocvar{For} \coqdocinductive{and}: \coqdocvar{Argument} \coqdocvar{scopes} \coqdocvar{are} [\coqdocvar{type\_scope} \coqdocvar{type\_scope}]\coqdoceol
\coqdocnoindent
\coqdocvar{For} \coqdocconstructor{conj}: \coqdocvar{Argument} \coqdocvar{scopes} \coqdocvar{are} [\coqdocvar{type\_scope} \coqdocvar{type\_scope} \coqdocvar{\_} \coqdocvar{\_}]

\coqdocemptyline


Proving a conjunction of \coqdocaxiom{P} and \coqdocaxiom{Q} therefore amounts to constructing
a pair by invoking the constructor \coqdocconstructor{conj} and providing values of \coqdocaxiom{P}
and \coqdocaxiom{Q} as its arguments:\footnote{The command
\texttt{Goal}\ccom{Goal} creates an anonymous theorem and initiates
the interactive proof mode.}


\begin{coqdoccode}
\coqdocemptyline
\coqdocnoindent
\coqdockw{Goal} \coqdocvar{P} \ensuremath{\rightarrow} \coqdocvar{$R$} \ensuremath{\rightarrow} \coqdocvar{P} \ensuremath{\land} \coqdocvar{$R$}.\coqdoceol
\coqdocnoindent
\coqdoctac{move}\ensuremath{\Rightarrow} \coqdocvar{p} \coqdocvar{r}.\coqdoceol
\coqdocemptyline
\end{coqdoccode}


The proof can be completed in several ways. The most familiar one is
to apply the constructor \coqdocconstructor{conj} directly. It will create two subgoals,
\coqdocaxiom{P} and \coqdocaxiom{Q} (which are the constructor arguments), that can be
immediately discharged.


\begin{coqdoccode}
\coqdocemptyline
\coqdocnoindent
\coqdoctac{apply}: \coqdocvar{conj}=>//.\coqdoceol
\coqdocemptyline
\end{coqdoccode}


Alternatively, since we now know that \coqdocinductive{and} has just one constructor,
we can use the generic Coq's \coqdoctac{constructor} \coqdocvar{n}\ttac{constructor}
tactic, where \coqdocvar{n} is an (optional) number of a constructor to be
applied (and in this case it's 1)


\begin{coqdoccode}
\coqdocemptyline
\coqdocnoindent
\coqdockw{Undo}.\coqdoceol
\coqdocnoindent
\coqdoctac{constructor} 1=>//.\coqdoceol
\coqdocemptyline
\end{coqdoccode}


Finally, for propositions that have exactly one constructor, Coq
provides a specialized tactic \coqdoctac{split}\ttac{split}, which is a synonym for
\coqdoctac{constructor} 1:
 \begin{coqdoccode}
\coqdocemptyline
\coqdocnoindent
\coqdockw{Undo}. \coqdoctac{split}=>//.\coqdoceol
\coqdocnoindent
\coqdockw{Qed}.\coqdoceol
\coqdocemptyline
\end{coqdoccode}


In order to prove something out of a conjunction, one needs to
\textit{destruct} it to get its constructor's arguments, and the simplest way
to do so is by the \coqdoctac{case}-analysis on a single constructor.


\begin{coqdoccode}
\coqdocemptyline
\coqdocnoindent
\coqdockw{Goal} \coqdocvar{P} \ensuremath{\land} \coqdocvar{Q} \ensuremath{\rightarrow} \coqdocvar{Q}.\coqdoceol
\coqdocnoindent
\coqdoctac{case}.\coqdoceol
\coqdocemptyline
\end{coqdoccode}


Again, the tactic \coqdoctac{case} replaced the top assumption \coqdocaxiom{P} \ensuremath{\land} \coqdocaxiom{Q} of the
goal with the arguments of its only constructor, \coqdocaxiom{P} and \coqdocaxiom{Q} making
the rest of the proof trivial.
\begin{coqdoccode}
\coqdocemptyline
\coqdocnoindent
\coqdocvar{\texttt{\emph{done}}}.\coqdoceol
\coqdocnoindent
\coqdockw{Qed}.\coqdoceol
\coqdocemptyline
\end{coqdoccode}
The datatype of disjunction of \coqdocaxiom{P} and \coqdocaxiom{Q}, denoted by \coqdocaxiom{P} \ensuremath{\lor} \coqdocaxiom{Q}, is
isomorphic to the \coqdocvar{sum} datatype from Chapter~\ref{ch:funprog} and
can be constructed by using one of its two constructors: \coqdocvar{or\_introl}
or \coqdocvar{or\_intror}.


\begin{coqdoccode}
\coqdocemptyline
\coqdocnoindent
\coqdockw{Locate} "\_ \symbol{92}/ \_".\coqdoceol
\coqdocemptyline
\end{coqdoccode}
"\coqdocvariable{A} \ensuremath{\lor} \coqdocvar{B}" := \coqdocinductive{or} \coqdocvariable{A} \coqdocvar{B}   : \coqdocvar{type\_scope} \begin{coqdoccode}
\coqdocemptyline
\coqdocnoindent
\coqdockw{Print} \coqdocvar{or}.\coqdoceol
\coqdocemptyline
\end{coqdoccode}


\coqdoceol
\coqdocemptyline
\coqdocnoindent
\coqdockw{Inductive} \coqdocinductive{or} (\coqdocvariable{A} \coqdocvar{B} : \coqdockw{Prop}) : \coqdockw{Prop} :=\coqdoceol
\coqdocindent{2.00em}
\coqdocvar{or\_introl} : \coqdocvariable{A} \ensuremath{\rightarrow} \coqdocvariable{A} \ensuremath{\lor} \coqdocvar{B} \ensuremath{|} \coqdocvar{or\_intror} : \coqdocvar{B} \ensuremath{\rightarrow} \coqdocvariable{A} \ensuremath{\lor} \coqdocvar{B}\coqdoceol
\coqdocnoindent
\coqdoceol
\coqdocnoindent
\coqdocvar{For} \coqdocvar{or\_introl}, \coqdocvar{when} \coqdocvar{applied} \coqdocvar{to} \coqdocvar{less} \coqdocvar{than} 1 \coqdocvar{argument}:\coqdoceol
\coqdocindent{1.00em}
\coqdocvar{Arguments} \coqdocvariable{A}, \coqdocvar{B} \coqdocvar{are} \coqdocvar{implicit}\coqdoceol
\coqdocnoindent
...

\coqdocemptyline


\ssrd{or}


In order to prove disjunction of \coqdocaxiom{P} and \coqdocaxiom{Q}, it is sufficient to
provide a proof of just \coqdocaxiom{P} or \coqdocaxiom{Q}, therefore appealing to the
appropriate constructor.


\begin{coqdoccode}
\coqdocemptyline
\coqdocnoindent
\coqdockw{Goal} \coqdocvar{Q} \ensuremath{\rightarrow} \coqdocvar{P} \ensuremath{\lor} \coqdocvar{Q} \ensuremath{\lor} \coqdocvar{$R$}.\coqdoceol
\coqdocnoindent
\coqdoctac{move}\ensuremath{\Rightarrow} \coqdocvar{q}.\coqdoceol
\coqdocemptyline
\end{coqdoccode}


Similarly to the case of conjunction, this proof can be completed
either by applying a constructor directly, by using \coqdoctac{constructor} 2
tactic or by a specialised Coq's tactic for disjunction:
\coqdoctac{left}\ttac{left}\ttac{right} or \coqdoctac{right}. The notation "\coqdocvar{\_} \ensuremath{\lor} \coqdocvar{\_}" is
right-associative, hence the following proof script, which first
reduces the goal to the proof of \coqdocaxiom{Q} \ensuremath{\lor} \coqdocdefinition{$R$}, and then to the proof of
\coqdocaxiom{Q}, which is trivial by assumption.


\begin{coqdoccode}
\coqdocemptyline
\coqdocnoindent
\coqdoctac{by} \coqdoctac{right}; \coqdoctac{left}.\coqdoceol
\coqdocnoindent
\coqdockw{Qed}.\coqdoceol
\coqdocemptyline
\end{coqdoccode}


The use of Ssreflect's tactical \coqdoctac{by}\ssrtl{by} makes sure that its
argument tactic (\coqdoctac{right}; \coqdoctac{left} in this case) succeeds and the proof of
the goal completes, similarly to the trailing \coqdocvar{\texttt{\emph{done}}}. If the sequence
of tactics \coqdoctac{left}; \coqdoctac{right} wouldn't prove the goal, a proof script error
would be reported.


The statements that have a disjunction as their assumption are usually
proved by case analysis on the two possible disjunction's
constructors:


\begin{coqdoccode}
\coqdocemptyline
\coqdocnoindent
\coqdockw{Goal} \coqdocvar{P} \ensuremath{\lor} \coqdocvar{Q} \ensuremath{\rightarrow} \coqdocvar{Q} \ensuremath{\lor} \coqdocvar{P}.\coqdoceol
\coqdocnoindent
\coqdoctac{case}\ensuremath{\Rightarrow}\coqdocvar{x}.\coqdoceol
\coqdocemptyline
\end{coqdoccode}


Notice how the case analysis via the Ssreflect's \coqdoctac{case} tactic was
combined here with the trailing \ensuremath{\Rightarrow}. It resulted in moving the
constructor parameter in \textit{each} of the subgoals from the goal
assumptions to the assumption context. The types of \coqdocvariable{x} are different
in the two branches of the proof, though. In the first branch, \coqdocvariable{x} has
type \coqdocaxiom{P}, as it names the argument of the \coqdocvar{or\_introl} constructor.


\coqdoceol
\coqdocemptyline
\coqdocindent{1.00em}
\coqdocaxiom{P} : \coqdockw{Prop}\coqdoceol
\coqdocindent{1.00em}
\coqdocaxiom{Q} : \coqdockw{Prop}\coqdoceol
\coqdocindent{1.00em}
\coqdocdefinition{$R$} : \coqdockw{Prop}\coqdoceol
\coqdocindent{1.00em}
\coqdocvariable{x} : \coqdocaxiom{P}\coqdoceol
\coqdocindent{1.00em}
============================\coqdoceol
\coqdocindent{1.50em}
\coqdocaxiom{Q} \ensuremath{\lor} \coqdocaxiom{P}\coqdoceol
\coqdocnoindent
\coqdoceol
\coqdocnoindent
\coqdockw{subgoal} 2 (\coqdocvar{ID} 248) \coqdockw{is}:\coqdoceol
\coqdocindent{0.50em}
\coqdocaxiom{Q} \ensuremath{\lor} \coqdocaxiom{P}

\coqdocemptyline
\begin{coqdoccode}
\coqdocemptyline
\coqdocnoindent
\coqdoctac{by} \coqdoctac{right}.\coqdoceol
\coqdocemptyline
\end{coqdoccode}


\coqdoceol
\coqdocemptyline
\coqdocindent{1.00em}
\coqdocaxiom{P} : \coqdockw{Prop}\coqdoceol
\coqdocindent{1.00em}
\coqdocaxiom{Q} : \coqdockw{Prop}\coqdoceol
\coqdocindent{1.00em}
\coqdocdefinition{$R$} : \coqdockw{Prop}\coqdoceol
\coqdocindent{1.00em}
\coqdocvariable{x} : \coqdocaxiom{Q}\coqdoceol
\coqdocindent{1.00em}
============================\coqdoceol
\coqdocindent{1.50em}
\coqdocaxiom{Q} \ensuremath{\lor} \coqdocaxiom{P}

\coqdocemptyline


In the second branch the type of \coqdocvariable{x} is \coqdocaxiom{Q}, as it accounts for the
case of the \coqdocvar{or\_intror} constructor.


\begin{coqdoccode}
\coqdocemptyline
\coqdocnoindent
\coqdoctac{by} \coqdoctac{left}.\coqdoceol
\coqdocnoindent
\coqdockw{Qed}.\coqdoceol
\coqdocemptyline
\end{coqdoccode}


It is worth noticing that the definition of disjunction in Coq is
\textit{constructive}, whereas the disjunction in classical propositional
logic is not. More precisely, in classical logic the proof of the
proposition \coqdocaxiom{P} \ensuremath{\lor} \textasciitilde \coqdocaxiom{P} is true by the axiom of excluded middle
(see Section~\ref{sec:axioms} for a more detailed discussion),
whereas in Coq, proving \coqdocaxiom{P} \ensuremath{\lor} \textasciitilde \coqdocaxiom{P} would amount to \textit{constructing} the
proof of either \coqdocaxiom{P} or \textasciitilde \coqdocaxiom{P}. Let us illustrate it with a specific
example. If \coqdocaxiom{P} is a proposition stating that \coqdocaxiom{P} = \coqdocvar{NP}, then in
classical logic tautology \coqdocaxiom{P} \ensuremath{\lor} \textasciitilde \coqdocaxiom{P} holds, although it does not
contribute to the proof of either of the disjuncts. In constructive
logic, which Coq is an implementation of, in the trivial assumptions
given the proof of \coqdocaxiom{P} \ensuremath{\lor} \textasciitilde \coqdocaxiom{P}, we would be able to extract the proof
of either \coqdocaxiom{P} or \textasciitilde\coqdocaxiom{P}.\footnote{Therefore, winning us the Clay
Institute's award.}




\section{Proofs with negation}




In Coq's constructive approach proving the negation of \textasciitilde\coqdocaxiom{P} of a
proposition \coqdocaxiom{P} literally means that one can derive
the falsehood from \coqdocaxiom{P}.


\begin{coqdoccode}
\coqdocemptyline
\coqdocnoindent
\coqdockw{Locate} "\~{} \_".\coqdoceol
\coqdocemptyline
\end{coqdoccode}


"\~{} \coqdocvariable{x}" := \coqdocdefinition{not} \coqdocvariable{x}       : \coqdocvar{type\_scope}
\begin{coqdoccode}
\coqdocemptyline
\coqdocnoindent
\coqdockw{Print} \coqdocvar{not}.\coqdoceol
\end{coqdoccode}


\coqdoceol
\coqdocemptyline
\coqdocindent{1.00em}
\coqdocdefinition{not} = \coqdockw{fun} \coqdocvariable{A} : \coqdockw{Prop} \ensuremath{\Rightarrow} \coqdocvariable{A} \ensuremath{\rightarrow} \coqdocinductive{False}\coqdoceol
\coqdocindent{3.50em}
: \coqdockw{Prop} \ensuremath{\rightarrow} \coqdockw{Prop}

\coqdocemptyline


Therefore, the negation \coqdocdefinition{not} on propositions from \coqdockw{Prop} is just a
function, which maps a proposition \coqdocvariable{A} to the implication \coqdocvariable{A} \ensuremath{\rightarrow}
\coqdocinductive{False}. With this respect the intuition of negation from classical
logic might be misleading: as it will be discussed in
Section~\ref{sec:axioms}, the Calculus of Constructions lacks the
double negation elimination axiom, which means that the proof of \textasciitilde \textasciitilde\coqdocvariable{A} will not
deliver the proof of \coqdocvariable{A}, as such derivation would be non-constructive,
as one cannot get a value of type \coqdocvariable{A} out of a function
of type (\coqdocvariable{A} \ensuremath{\rightarrow} \coqdocvar{B}) \ensuremath{\rightarrow} \coqdocvar{B}, where \coqdocvar{B} is taken to be \coqdocinductive{False}.


However, reasoning out of negation helps to derive the familiar proofs
by contradiction, given that we managed to construct \coqdocaxiom{P} \textit{and} \textasciitilde\coqdocaxiom{P},
as demonstrated by the following theorem, which states that from any
\coqdocaxiom{Q} can be derived from \coqdocaxiom{P} and \textasciitilde\coqdocaxiom{P}. 


\begin{coqdoccode}
\coqdocemptyline
\coqdocnoindent
\coqdockw{Theorem} \coqdocvar{absurd}: \coqdocvar{P} \ensuremath{\rightarrow} \textasciitilde\coqdocvar{P} \ensuremath{\rightarrow} \coqdocvar{Q}.\coqdoceol
\coqdocnoindent
\coqdockw{Proof}. \coqdoctac{by} \coqdoctac{move}\ensuremath{\Rightarrow}\coqdocvar{p} \coqdocvar{H}; \coqdoctac{move} : (\coqdocvar{H} \coqdocvar{p}). \coqdockw{Qed}.\coqdoceol
\coqdocemptyline
\end{coqdoccode}


One extremely useful theorem from propositional logic involving
negation is \textit{contraposition}. It states that in an implication, the
assumption and the goal can be flipped if inverted.


\begin{coqdoccode}
\coqdocemptyline
\coqdocnoindent
\coqdockw{Theorem} \coqdocvar{contraP}: (\coqdocvar{P} \ensuremath{\rightarrow} \coqdocvar{Q}) \ensuremath{\rightarrow} \textasciitilde\coqdocvar{Q} \ensuremath{\rightarrow} \textasciitilde\coqdocvar{P}.\coqdoceol
\coqdocemptyline
\end{coqdoccode}
Let us see how it can be proved in Coq \begin{coqdoccode}
\coqdocemptyline
\coqdocnoindent
\coqdockw{Proof}.\coqdoceol
\coqdocnoindent
\coqdoctac{move}\ensuremath{\Rightarrow} \coqdocvar{H} \coqdocvar{Hq}.\coqdoceol
\coqdocnoindent
\coqdoctac{move} /\coqdocvar{H}.\coqdoceol
\end{coqdoccode}
\coqdoceol
\coqdocemptyline
\coqdocindent{1.00em}
\coqdocvar{H} : \coqdocaxiom{P} \ensuremath{\rightarrow} \coqdocaxiom{Q}\coqdoceol
\coqdocindent{1.00em}
\coqdocvar{Hq} : \textasciitilde \coqdocaxiom{Q}\coqdoceol
\coqdocindent{1.00em}
============================\coqdoceol
\coqdocindent{1.50em}
\coqdocaxiom{Q} \ensuremath{\rightarrow} \coqdocinductive{False}

\coqdocemptyline


The syntax \coqdoctac{move} / \coqdocvar{H} (spaces in between are optional) stands for one
of the most powerful features of Ssreflect, called \textit{views} (see
Section~9 of~\cite{Gontier-al:TR}), which allows one to \textit{weaken} the
assumption in the goal part of the proof on the fly by applying a
hypothesis \coqdocvar{H} to the top assumption in the goal. In the script above
the first command \coqdoctac{move}\ensuremath{\Rightarrow} \coqdocvar{H} \coqdocvar{Hq} simply popped two assumptions from the
goal to the context. What is left is \textasciitilde\coqdocaxiom{P}, or, equivalently \coqdocaxiom{P} \ensuremath{\rightarrow}
\coqdocinductive{False}. The view mechanism then ``interpreted'' \coqdocaxiom{P} in the goal via \coqdocvar{H}
and changing it to \coqdocaxiom{Q}, since \coqdocvar{H} was of type \coqdocaxiom{P} \ensuremath{\rightarrow} \coqdocaxiom{Q}, which results
in the modified goal \coqdocaxiom{Q} \ensuremath{\rightarrow} \coqdocinductive{False}.  Next, we apply the view \coqdocvar{Hq} to
the goal, which switches \coqdocaxiom{Q} to \coqdocinductive{False}, which makes the rest of the
proof trivial.  \begin{coqdoccode}
\coqdocemptyline
\coqdocnoindent
\coqdoctac{move} /\coqdocvar{Hq}.\coqdoceol
\coqdocnoindent
\coqdocvar{\texttt{\emph{done}}}.\coqdoceol
\coqdocnoindent
\coqdockw{Qed}.\coqdoceol
\coqdocemptyline
\coqdocemptyline
\coqdocemptyline
\end{coqdoccode}
\section{Existential quantification}


\label{sec:exists}


Existential quantification in Coq, which is denoted by the notation
``\coqdoctac{exists} \coqdocvariable{x}, \coqdocaxiom{P} \coqdocvariable{x}'' is just yet another inductive predicate with exactly
one constructor:


\begin{coqdoccode}
\coqdocemptyline
\coqdocnoindent
\coqdockw{Locate} "exists".\coqdoceol
\coqdocemptyline
\end{coqdoccode}
\coqdoceol
\coqdocemptyline
\coqdocnoindent
"'exists' x .. y , p" := \coqdocinductive{ex} (\coqdockw{fun} \coqdocvariable{x} \ensuremath{\Rightarrow} .. (\coqdocinductive{ex} (\coqdockw{fun} \coqdocvariable{y} \ensuremath{\Rightarrow} \coqdocvar{p})) ..)\coqdoceol
\coqdocindent{11.00em}
: \coqdocvar{type\_scope}

\coqdocemptyline


\begin{coqdoccode}
\coqdocemptyline
\coqdocnoindent
\coqdockw{Print} \coqdocvar{ex}.\coqdoceol
\coqdocemptyline
\end{coqdoccode}
\coqdoceol
\coqdocemptyline
\coqdocnoindent
\coqdockw{Inductive} \coqdocinductive{ex} (\coqdocvariable{A} : \coqdockw{Type}) (\coqdocaxiom{P} : \coqdocvariable{A} \ensuremath{\rightarrow} \coqdockw{Prop}) : \coqdockw{Prop} :=\coqdoceol
\coqdocindent{2.00em}
\coqdocvar{ex\_intro} : \coqdockw{\ensuremath{\forall}} \coqdocvariable{x} : \coqdocvariable{A}, \coqdocaxiom{P} \coqdocvariable{x} \ensuremath{\rightarrow} \coqdocinductive{ex} \coqdocvariable{A} \coqdocaxiom{P}

\coqdocemptyline


\ssrd{ex}


The notation for existentially quantified predicates conveniently
allows one to existentially quantify over several variables,
therefore, leading to a chain of enclosed calls of the constructor
\coqdocvar{ex\_intro}.  


The inductive predicate \coqdocinductive{ex} is parametrized with a type \coqdocvariable{A}, over
elements of which we quantify, and a predicate function of type \coqdocvariable{A} \ensuremath{\rightarrow}
\coqdockw{Prop}. What is very important is that the scope of the variable \coqdocvariable{x} in
the constructor captures \coqdocinductive{ex} \coqdocvariable{A} \coqdocaxiom{P} as well. That is, the constructor
type could be written as \coqdockw{\ensuremath{\forall}} \coqdocvariable{x}, (\coqdocaxiom{P} \coqdocvariable{x} \ensuremath{\rightarrow} \coqdocinductive{ex} \coqdocvariable{A} \coqdocaxiom{P}) to emphasize that
each particular instance of \coqdocinductive{ex} \coqdocvariable{A} \coqdocaxiom{P} carries is defined by a
\textit{particular} value of \coqdocvariable{x}. The actual value of \coqdocvariable{x}, which satisfies
the predicate \coqdocaxiom{P} is, however, not exposed to the client, providing
the \textit{data abstraction} and information hiding, similarly to the
traditional existential types (see Chapter~24
of~\cite{Pierce:BOOK02}), which would serve as a good analogy.  Each
inhabitant of the type \coqdocinductive{ex} is therefore an instance of a
\emph{dependent pair},\footnote{In the literature, dependent pairs
are often referred to as \emph{dependent sums} or
$\Sigma$-types.\index{dependent sum}} whose first element is a
\textit{witness} for the following predicate \coqdocaxiom{P}, and the second one is a
result of application of \coqdocaxiom{P} to \coqdocvariable{x}, yielding a particular
proposition.


\index{Sigma-type|see {dependent sum}}


The proofs of propositions that assume existential quantification are
simply the proofs by case analysis: destructing the only constructor
of \coqdocinductive{ex}, immediately provides its arguments: a witness \coqdocvariable{x} and the
predicate \coqdocaxiom{P}, satisfied by \coqdocvariable{x}. The proofs, where the existential
quantification is a goal, can be completed by applying the constructor
\coqdocvar{ex\_intro} directly or by using a specialized Coq's tactic \coqdoctac{exists} \coqdocvariable{z},
which does exactly the same, instantiating the first parameter of the
constructor with the provided value \coqdocvariable{z}. Let us demonstrate it on a
simple example~\cite[\S 5.2.6]{Bertot-Casteran:BOOK}, accounting for
the weakening of the predicate, satisfying the existentially
quantified variable.


\begin{coqdoccode}
\coqdocemptyline
\coqdocnoindent
\coqdockw{Theorem} \coqdocvar{ex\_imp\_ex} \coqdocvar{A} (\coqdocvar{S} \coqdocvar{T}: \coqdocvar{A} \ensuremath{\rightarrow} \coqdockw{Prop}): \coqdoceol
\coqdocindent{1.00em}
(\coqdoctac{exists} \coqdocvar{a}: \coqdocvar{A}, \coqdocvar{S} \coqdocvar{a}) \ensuremath{\rightarrow} (\coqdockw{\ensuremath{\forall}} \coqdocvar{x}: \coqdocvar{A}, \coqdocvar{S} \coqdocvar{x} \ensuremath{\rightarrow} \coqdocvar{T} \coqdocvar{x}) \ensuremath{\rightarrow} \coqdoctac{exists} \coqdocvar{b}: \coqdocvar{A}, \coqdocvar{T} \coqdocvar{b}.\coqdoceol
\coqdocemptyline
\end{coqdoccode}


The parentheses are important here, otherwise, for instance, the scope
of the first existentially-quantified variable \coqdocvariable{a} would be the whole
subsequent statement, not just the proposition \textit{S a}.


\begin{coqdoccode}
\coqdocemptyline
\coqdocnoindent
\coqdockw{Proof}.\coqdoceol
\coqdocemptyline
\end{coqdoccode}


First, we decompose the first existential product into the witness \coqdocvariable{a}
and the proposition \coqdocvar{Hst}, and also store the universally-quantified
implication assumption with the name \coqdocvar{Hst}.


\begin{coqdoccode}
\coqdocemptyline
\coqdocnoindent
\coqdoctac{case}\ensuremath{\Rightarrow}\coqdocvar{a} \coqdocvar{Hs} \coqdocvar{Hst}.\coqdoceol
\end{coqdoccode}


\coqdoceol
\coqdocemptyline
\coqdocindent{1.00em}
\coqdocvariable{A} : \coqdockw{Type}\coqdoceol
\coqdocindent{1.00em}
\coqdocdefinition{S} : \coqdocvariable{A} \ensuremath{\rightarrow} \coqdockw{Prop}\coqdoceol
\coqdocindent{1.00em}
\coqdocvariable{T} : \coqdocvariable{A} \ensuremath{\rightarrow} \coqdockw{Prop}\coqdoceol
\coqdocindent{1.00em}
\coqdocvariable{a} : \coqdocvariable{A}\coqdoceol
\coqdocindent{1.00em}
\coqdocvar{Hs} : \coqdocdefinition{S} \coqdocvariable{a}\coqdoceol
\coqdocindent{1.00em}
\coqdocvar{Hst} : \coqdockw{\ensuremath{\forall}} \coqdocvariable{x} : \coqdocvariable{A}, \coqdocdefinition{S} \coqdocvariable{x} \ensuremath{\rightarrow} \coqdocvariable{T} \coqdocvariable{x}\coqdoceol
\coqdocindent{1.00em}
============================\coqdoceol
\coqdocindent{1.50em}
\coqdoctac{exists} \coqdocvariable{b} : \coqdocvariable{A}, \coqdocvariable{T} \coqdocvariable{b}

\coqdocemptyline
Next, we apply the \coqdocinductive{ex}'s constructor by means of the \coqdoctac{exists}
tactic with an explicit witness value \coqdocvariable{a}: 


\ssrt{exists}


\begin{coqdoccode}
\coqdocemptyline
\coqdocnoindent
\coqdoctac{exists} \coqdocvar{a}.\coqdoceol
\coqdocemptyline
\end{coqdoccode}
We finish the proof  by applying the weakening hypothesis \coqdocvar{Hst}. \begin{coqdoccode}
\coqdocemptyline
\coqdocnoindent
\coqdoctac{by} \coqdoctac{apply}: \coqdocvar{Hst}.\coqdoceol
\coqdocemptyline
\coqdocnoindent
\coqdockw{Qed}.\coqdoceol
\coqdocemptyline
\end{coqdoccode}
\begin{exercise}


Let us define our own version \coqdocinductive{my\_ex} of the existential quantifier
using the Ssreflect notation for constructors:


\begin{coqdoccode}
\coqdocemptyline
\coqdocnoindent
\coqdockw{Inductive} \coqdocvar{my\_ex} \coqdocvar{A} (\coqdocvar{S}: \coqdocvar{A} \ensuremath{\rightarrow} \coqdockw{Prop}) : \coqdockw{Prop} := \coqdocvar{my\_ex\_intro} \coqdocvar{x} \coqdockw{\texttt{\emph{of}}} \coqdocvar{S} \coqdocvar{x}.\coqdoceol
\coqdocemptyline
\end{coqdoccode}


The reader is invited to prove the following goal, establishing the
equivalence of the two propositions


\begin{coqdoccode}
\coqdocemptyline
\coqdocnoindent
\coqdockw{Goal} \coqdockw{\ensuremath{\forall}} \coqdocvar{A} (\coqdocvar{S}: \coqdocvar{A} \ensuremath{\rightarrow} \coqdockw{Prop}), \coqdocvar{my\_ex} \coqdocvar{A} \coqdocvar{S} \texttt{<->} \coqdoctac{exists} \coqdocvar{y}: \coqdocvar{A}, \coqdocvar{S} \coqdocvar{y}.\coqdoceol
\coqdocemptyline
\end{coqdoccode}


\hint the propositional equivalence \texttt{<->} is just a conjunction of
two implications, so proving it can be reduced to two separate goals
by means of \coqdoctac{split} tactics.


\begin{coqdoccode}
\coqdocemptyline
\coqdocemptyline
\end{coqdoccode}


\end{exercise}


\subsection{A conjunction and disjunction analogy}




Sometimes, the universal and the existential quantifications are
paraphrased as ``infinitary'' conjunction and disjunction
correspondingly. This analogy comes in handy when understanding the
properties of both quantifications, so let us elaborate on it for a bit.


In order to prove the conjunction \coqdocvar{P1} \ensuremath{\land} ... \ensuremath{\land} \coqdocvar{Pn}, one needs to
establish that \textit{all} propositions \coqdocvar{P1} ... \coqdocvar{Pn} hold, which in the
finite case can be done by proving \coqdocvar{n} goals, for each statement
separately (and this is what the \coqdoctac{split} tactic helps to
do). Similarly, in order to prove the propositions \coqdockw{\ensuremath{\forall}} \coqdocvariable{x}: \coqdocvariable{A}, \coqdocaxiom{P} \coqdocvariable{x},
one needs to prove that \coqdocaxiom{P} \coqdocvariable{x} holds for \textit{any} \coqdocvariable{x} of type \coqdocvariable{A}. Since
the type \coqdocvariable{A} itself can define an infinite set, there is no way to
enumerate all conjuncts, however, an explicit handle \coqdocvariable{x} gives a way
to effectively \textit{index} them, so proving \coqdocaxiom{P} \coqdocvariable{x} for an arbitrary \coqdocvariable{x} would
establish the validity of the universal quantification itself. Another
useful insight is that in Coq \coqdockw{\ensuremath{\forall}} \coqdocvariable{x}: \coqdocvariable{A}, \coqdocaxiom{P} \coqdocvariable{x} is a type of a
dependent function that maps \coqdocvariable{x} of type \coqdocvariable{A} to a value of type \coqdocaxiom{P}
\coqdocvariable{x}. The proof of the quantification would be, therefore, a function
with a suitable ``plot''. Similarly, in the case of \coqdocvar{n}-ary conjunction,
the function has finite domain of exactly \coqdocvar{n} points, for each of
which an appropriate proof term should be found.


In order to prove the \coqdocvar{n}-ary disjunction \coqdocvar{P1} \ensuremath{\lor} ... \ensuremath{\lor} \coqdocvar{Pn} in Coq, it
is sufficient to provide a proof for just one of the disjunct \textit{as well
as} a ``tag'' --- an indicator, which disjunct exactly is being proven
(this is what tactics \coqdoctac{left} and \coqdoctac{right} help to achieve). In the case
of infinitary disjunction, the existential quantification ``\coqdoctac{exists} \coqdocvariable{x},
\coqdocaxiom{P} \coqdocvariable{x}'', the existentially quantified variable plays role of the tag
indexing all possible propositions \coqdocaxiom{P} \coqdocvariable{x}. Therefore, in order to prove
such a proposition, one needs first to deliver a witness \coqdocvariable{x} (usually,
by means of calling the tactics \coqdoctac{exists}), and then prove that for
this witness/tag the proposition \coqdocaxiom{P} \coqdocvariable{x} holds. Continuing the same
analogy, the disjunction in the assumption of a goal usually leads to
the proof by \coqdoctac{case} analysis assuming that one of the disjuncts holds
at a time. Similarly, the way to destruct the existential
quantification is by case-analysing on its constructor, which results
in acquiring the witness (i.e., the ``tag'') and the corresponding
``disjunct''.


Finally, the folklore alias ``dependent product type'' for dependent
function types (i.e., \coqdockw{\ensuremath{\forall}}-quantified types) indicates its
relation to products, which are Curry-Howard counterparts of
conjunctions. In the same spirit, the term ``dependent sum type'' for
the dependent types, of which existential quantification is a
particular case, hints to the relation to the usual sum types, and, in
particular \coqdocvar{sum} (discussed in Chapter~\ref{ch:funprog}), which is a
Curry-Howard dual of a disjunction.


\begin{coqdoccode}
\coqdocemptyline
\coqdocnoindent
\coqdockw{End} \coqdocvar{Connectives}.\coqdoceol
\coqdocemptyline
\end{coqdoccode}
\section{Missing axioms from classical logic}


\label{sec:axioms}


In the previous sections of this chapter, we have seen how a fair
amount of propositions from the higher-order propositional logics can
be encoded and proved in Coq. However, some reasoning principles,
employed in \textit{classical} propositional logic, cannot be encoded in
Coq in a form of provable statements, and, hence, should be encoded as
\textit{axioms}.


\index{classical propositional logic} In this section, we provide a
brief and by all means incomplete overview of the classical
propositional logic axioms that are missing in Coq, but can be added
by means of importing the appropriate libraries. Chapter~12 of the
book~\cite{Chlipala:BOOK} contains a detailed survey of useful axioms
that can be added into Coq development on top of CIC.


To explore some of some of the axioms, we first import that classical
logic module \coqdoclibrary{Classical\_Prop}.


\begin{coqdoccode}
\coqdocemptyline
\coqdocemptyline
\coqdocnoindent
\coqdockw{Import} \coqdocvar{Classical\_Prop}.\coqdoceol
\coqdocemptyline
\end{coqdoccode}


The most often missed axiom is the axiom of \textit{excluded middle}, which
postulates that the disjunction of \coqdocaxiom{P} and \textasciitilde\coqdocaxiom{P} is provable. Adding
this axiom circumvents the fact that the reasoning out of the excluded
middle principle is \textit{non-constructive}, as discussed in
Section~\ref{sec:conjdisj}.


\begin{coqdoccode}
\coqdocemptyline
\coqdocnoindent
\coqdockw{Check} \coqdocvar{classic}.\coqdoceol
\coqdocemptyline
\end{coqdoccode}


\coqdoceol
\coqdocemptyline
\coqdocnoindent
\coqdocaxiom{classic}\coqdoceol
\coqdocindent{2.50em}
: \coqdockw{\ensuremath{\forall}} \coqdocaxiom{P} : \coqdockw{Prop}, \coqdocaxiom{P} \ensuremath{\lor} \textasciitilde \coqdocaxiom{P}

\coqdocemptyline


Another axiom from the classical logic, which coincides with the type
of Scheme's \coqdocvar{call}/\coqdocvar{cc}
operator\footnote{\url{http://community.schemewiki.org/?call-with-current-continuation}}
(pronounced as \textit{call with current continuation}) modulo Curry-Howard
isomorphism is \textit{Peirce's law}:
\index{Peirce's law}
\begin{coqdoccode}
\coqdocemptyline
\coqdocnoindent
\coqdockw{Definition} \coqdocvar{peirce\_law} := \coqdockw{\ensuremath{\forall}} \coqdocvar{P} \coqdocvar{Q}: \coqdockw{Prop}, ((\coqdocvar{P} \ensuremath{\rightarrow} \coqdocvar{Q}) \ensuremath{\rightarrow} \coqdocvar{P}) \ensuremath{\rightarrow} \coqdocvar{P}.\coqdoceol
\coqdocemptyline
\end{coqdoccode}


In Scheme-like languages, the \coqdocvar{call}/\coqdocvar{cc} operator allows one to
invoke the undelimited continuation, which aborts the
computation. Similarly to the fact that \coqdocvar{call}/\coqdocvar{cc} cannot be
implemented in terms of polymorphically-typed lambda calculus as a
function and should be added as an external operator, the Peirce's law
is an axiom in the constructive logic.


The classical double negation principle is easily derivable from
Peirce's law, and corresponds to the type of \coqdocvar{call}/\coqdocvar{cc}, which always
invokes its continuation parameter, aborting the current computation.


\begin{coqdoccode}
\coqdocemptyline
\coqdocnoindent
\coqdockw{Check} \coqdocvar{NNPP}.\coqdoceol
\coqdocemptyline
\end{coqdoccode}


\coqdoceol
\coqdocemptyline
\coqdocnoindent
\coqdoclemma{NNPP}\coqdoceol
\coqdocindent{2.50em}
: \coqdockw{\ensuremath{\forall}} \coqdocaxiom{P} : \coqdockw{Prop}, \textasciitilde \textasciitilde\coqdocaxiom{P} \ensuremath{\rightarrow} \coqdocaxiom{P}

\coqdocemptyline


Finally, the classical formulation of the implication through the
disjunction is again an axiom in the constructive logic, as otherwise
from the function of type \coqdocaxiom{P} \ensuremath{\rightarrow} \coqdocaxiom{Q} one would be able to construct the
proof of \textasciitilde\coqdocaxiom{P} \ensuremath{\lor} \coqdocaxiom{Q}, which would make the law of excluded middle
trivial to derive.
\begin{coqdoccode}
\coqdocemptyline
\coqdocnoindent
\coqdockw{Check} \coqdocvar{imply\_to\_or}.\coqdoceol
\coqdocemptyline
\end{coqdoccode}


\coqdoceol
\coqdocemptyline
\coqdocnoindent
\coqdoclemma{imply\_to\_or}\coqdoceol
\coqdocindent{2.50em}
: \coqdockw{\ensuremath{\forall}} \coqdocaxiom{P} \coqdocaxiom{Q} : \coqdockw{Prop}, (\coqdocaxiom{P} \ensuremath{\rightarrow} \coqdocaxiom{Q}) \ensuremath{\rightarrow} \textasciitilde\coqdocaxiom{P} \ensuremath{\lor} \coqdocaxiom{Q}

\coqdocemptyline


Curiously, none of these axioms, if added to Coq, makes its logic
unsound: it has been rigorously proven (although, not within Coq, due
to \Godel's incompleteness result) that all classical logic axioms
are consistent with CIC, and, therefore, don't make it possible to
derive the falsehood~\cite{Werner:TACS97}.


The following exercise reconciles most of the familiar axioms of 
classical logic.


\begin{exercise}[Equivalence of classical logic axioms (from \S~5.2.4
of~\cite{Bertot-Casteran:BOOK})] \label{ex:equivax} 


Prove that the following five axioms of the classical are equivalent.


\begin{coqdoccode}
\coqdocemptyline
\coqdocnoindent
\coqdockw{Definition} \coqdocvar{peirce} := \coqdocvar{peirce\_law}.\coqdoceol
\coqdocnoindent
\coqdockw{Definition} \coqdocvar{double\_neg} := \coqdockw{\ensuremath{\forall}} \coqdocvar{P}: \coqdockw{Prop}, \textasciitilde \textasciitilde \coqdocvar{P} \ensuremath{\rightarrow} \coqdocvar{P}.\coqdoceol
\coqdocnoindent
\coqdockw{Definition} \coqdocvar{excluded\_middle} := \coqdockw{\ensuremath{\forall}} \coqdocvar{P}: \coqdockw{Prop}, \coqdocvar{P} \ensuremath{\lor} \textasciitilde\coqdocvar{P}.\coqdoceol
\coqdocnoindent
\coqdockw{Definition} \coqdocvar{de\_morgan\_not\_and\_not} := \coqdockw{\ensuremath{\forall}} \coqdocvar{P} \coqdocvar{Q}: \coqdockw{Prop}, \textasciitilde ( \textasciitilde\coqdocvar{P} \ensuremath{\land} \textasciitilde\coqdocvar{Q}) \ensuremath{\rightarrow} \coqdocvar{P} \ensuremath{\lor} \coqdocvar{Q}.\coqdoceol
\coqdocnoindent
\coqdockw{Definition} \coqdocvar{implies\_to\_or} := \coqdockw{\ensuremath{\forall}} \coqdocvar{P} \coqdocvar{Q}: \coqdockw{Prop}, (\coqdocvar{P} \ensuremath{\rightarrow} \coqdocvar{Q}) \ensuremath{\rightarrow} (\~{}\coqdocvar{P} \ensuremath{\lor} \coqdocvar{Q}).\coqdoceol
\coqdocemptyline
\coqdocemptyline
\end{coqdoccode}


\hint Use \coqdoctac{rewrite} /\coqdocvar{d} \ssrt{rewrite} tactics to unfold the
 definition of a value \coqdocvar{d} and replace its name by its body. You can
 chain several unfoldings by writing \coqdoctac{rewrite} /\coqdocvar{d1} /\coqdocvar{d2} ...
 etc. \ssrt{rewrite}


\hint To facilitate the forward reasoning by contradiction, you can
 use the Ssreflect tactic \coqdocvar{\texttt{\emph{suff}}}: \coqdocaxiom{P}, \ssrt{suff:} where \coqdocaxiom{P} is
 an arbitrary proposition. The system will then require you to prove
 that \coqdocaxiom{P} implies the goal \textit{and} \coqdocaxiom{P} itself.


\ccom{Admitted}


\hint Stuck with a tricky proof? Use the Coq \coqdocvar{Admitted} keyword as a
 ``stub'' for an unfinished proof of a goal, which, nevertheless will be
 considered completed by Coq. You can always get back to an admitted
 proof later.


\end{exercise}




\section{Universes and \texorpdfstring{\protect\coqdockw{Prop}}{Prop} impredicativity}




While solving Exercise~\ref{ex:equivax} from the previous section,
the reader could notice an interesting detail about the propositions
in Coq and the sort \coqdockw{Prop}: the propositions that quantify over
propositions still remain to be propositions, i.e., they still belong
to the sort \coqdockw{Prop}. This property of propositions in Coq (and, in
general, the ability of entities of some class to abstract over the
entities of the same class) is called
\textit{impredicativity}. \index{impredicativity} The opposite
characteristic (i.e., the inability to refer to the elements of the
same class) is called \textit{predicativity}. \index{predicativity}


One of the main challenges when designing the Calculus of
Constructions was to implement its logical component (i.e., the
fragment responsible for constructing and operating with elements of
the \coqdockw{Prop} sort), so it would subsume the existing impredicative
propositional calculi~\cite{Coquand-Huet:ECCA85}, and, in
\index{System $F$}
particular, System~$F$ (which is impredicative), allowing for the
expressive reasoning in higher-order propositional logic.


\textit{Impredicativity} as a property of definitions allows one to define
domains that are \textit{self-recursive}---a feature of \coqdockw{Prop} that we
recently observed. Unfortunately, when restated in the classical set
theory, impredicativity immediately leads to the famous paradox by Russell,
which arises from the attempt to define the set of all sets
that do not belong to themselves. In terms of programming,
Russell's paradox provides a recipe to encode a fixpoint combinator in
the calculus itself and write generally-recursive programs.


System~$F$ is not a dependently-typed calculus and it has been
proven to contain no paradoxes~\cite{Girard:PhD}, as it reasons only
about \textit{types} (or, \textit{propositions}), which do not depend on
values. However, adding dependent types to the mix (which Coq requires
to make propositions quantify over arbitrary values, not just other
propositions, serving as a general-purpose logic) makes the design of
a calculus more complicated, in order to avoid paradoxes akin to the
Russell's, which arise from mixing values and sets of values. This
necessity to ``cut the knot'' inevitably requires to have a sort of a
higher level, which contains all sets and propositions (i.e., the
whole sorts \coqdockw{Set} and \coqdockw{Prop}), but does not contain itself. Let us
call such sort \coqdockw{Type}. It turns out that the self-inclusion \coqdockw{Type} :
\coqdockw{Type} leads to another class of paradoxes~\cite{Coquand:LICS86}, and
in order to avoid them, the hierarchy of higher-order sorts should be
made infinite and \textit{stratified}. \index{stratification}
Stratification means that each sort has a level number, and is
contained in a sort of a higher level but not in itself. The described
approach is suggested by Martin-\loef~\cite{Martin-Loef:84} and
adopted literally, in particular, by
Agda~\cite{Norell:PhD}\index{Agda}. The stratified type sorts,
following Martin-\loef's tradition, are usually referred to as
\textit{universes}.\index{universes}


A similar stratification is also implemented in Coq, which has its own
universe hierarchy, although with an important twist. The two
universes, \coqdockw{Set} and \coqdockw{Prop} cohabit at the first level of the universe
hierarchy with \coqdockw{Prop} being impredicative. The universe containing
both \coqdockw{Set} and \coqdockw{Prop} is called \coqdockw{Type}@\{\coqdockw{Set}+1\}, and it is \textit{predicative}, as
well as all universes that are above it in the hierarchy. CIC
therefore remains consistent as a calculus, only because of the fact
that all impredicativity in it is contained at the very bottom of the
hierarchy.


\subsection{Exploring and debugging the universe hierarchy}




In the light of Martin-\loef's stratification, the Coq'
non-polymorphic types, such as \coqdocvar{nat}, \coqdocinductive{bool}, \coqdocvar{unit} or \coqdocinductive{list} \coqdocvar{nat}
``live'' at the 0th level of universe hierarchy, namely, in the sort
\coqdockw{Set}. The polymorphic types, quantifying over the elements of the
\coqdockw{Set} universe are, therefore located at the higher level, which in
Coq is denoted as \coqdockw{Type}@\{\coqdockw{Set}+1\}, but in the displays is usually presented
simply as \coqdockw{Type}, as well as all the higher universes. We can enable
the explicit printing of the universe levels to see how they are
assigned:


\ccom{Set Printing Universes}


\begin{coqdoccode}
\coqdocemptyline
\coqdocnoindent
\coqdockw{Set Printing Universes}.\coqdoceol
\coqdocemptyline
\coqdocnoindent
\coqdockw{Check} \coqdocvar{bool}.\coqdoceol
\coqdocemptyline
\end{coqdoccode}
\coqdoceol
\coqdocemptyline
\coqdocnoindent
\coqdocinductive{bool}\coqdoceol
\coqdocindent{2.50em}
: \coqdockw{Set}

\coqdocemptyline
\begin{coqdoccode}
\coqdocemptyline
\coqdocnoindent
\coqdockw{Check} \coqdockw{Set}.\coqdoceol
\coqdocemptyline
\end{coqdoccode}
\coqdoceol
\coqdocemptyline
\coqdocnoindent
\coqdockw{Set}\coqdoceol
\coqdocindent{2.50em}
: \coqdocvar{Type@}\{\coqdockw{Set}+1\}

\coqdocemptyline
\begin{coqdoccode}
\coqdocemptyline
\coqdocnoindent
\coqdockw{Check} \coqdockw{Prop}.\coqdoceol
\coqdocemptyline
\end{coqdoccode}
\coqdoceol
\coqdocemptyline
\coqdocnoindent
\coqdockw{Prop}\coqdoceol
\coqdocindent{2.50em}
: \coqdocvar{Type@}\{\coqdockw{Set}+1\}

\coqdocemptyline


The following type is polymorphic over the elements of the \coqdockw{Set}
universe, and this is why its own universe is ``bumped up'' by one, so
it now belongs to \coqdockw{Set}+1.


\begin{coqdoccode}
\coqdocemptyline
\coqdocnoindent
\coqdockw{Definition} \coqdocvar{S} := \coqdockw{\ensuremath{\forall}} \coqdocvar{T}: \coqdockw{Set}, \coqdocvar{list} \coqdocvar{T}.\coqdoceol
\coqdocnoindent
\coqdockw{Check} \coqdocvar{S}.\coqdoceol
\coqdocemptyline
\end{coqdoccode}


\coqdoceol
\coqdocemptyline
\coqdocnoindent
\coqdocdefinition{S}\coqdoceol
\coqdocindent{2.50em}
: \coqdocvar{Type@}\{\coqdockw{Set}+1\}

\coqdocemptyline


\index{universe polymorphism} 


Until version 8.5, Coq used to provide a very limited version of
\textit{universe polymorphism}. To illustrate the idea, let us consider the
next definition \coqdocdefinition{$R$}, which is polymorphic with respect to the universe
of its parameter \coqdocvariable{A}, so its result is assumed to ``live'' in the
universe, whose level is taken to be the level of \coqdocvariable{A}.


\begin{coqdoccode}
\coqdocemptyline
\coqdocnoindent
\coqdockw{Definition} \coqdocvar{$R$} (\coqdocvar{A}: \coqdockw{Type}) (\coqdocvar{x}: \coqdocvar{A}): \coqdocvar{A} := \coqdocvar{x}.\coqdoceol
\coqdocnoindent
\coqdocvar{Arguments} \coqdocvar{$R$} [\coqdocvar{A}].\coqdoceol
\coqdocemptyline
\coqdocnoindent
\coqdockw{Check} \coqdocvar{$R$} \coqdocvar{tt}.\coqdoceol
\coqdocemptyline
\end{coqdoccode}


\coqdoceol
\coqdocemptyline
\coqdocnoindent
\coqdocdefinition{$R$} \coqdocconstructor{tt}\coqdoceol
\coqdocindent{2.50em}
: \coqdocvar{unit}

\coqdocemptyline

\texttt{\coqcomm{ |= Set <= Top.93 }}



The part in comments show the inequality, generated by the Coq
unification algorithm that had to be solved in order to determine the
universe level of the value \coqdocdefinition{$R$} \coqdocconstructor{tt} with \coqdocvar{Top}.93 being the level,
assigned to \coqdocdefinition{$R$} itself.






If the argument of \coqdocdefinition{$R$} is itself a universe, it means that \coqdocvariable{A}'s
level is higher than \coqdocvariable{x}'s level, and so is the level of \coqdocdefinition{$R$}'s result.


\begin{coqdoccode}
\coqdocemptyline
\coqdocnoindent
\coqdockw{Check} \coqdocvar{$R$} \coqdockw{Type}.\coqdoceol
\coqdocemptyline
\end{coqdoccode}


\coqdoceol
\coqdocemptyline
\coqdocnoindent
\coqdocdefinition{$R$} \coqdocvar{Type@}\{\coqdocvar{Top}.94\}\coqdoceol
\coqdocindent{2.50em}
: \coqdocvar{Type@}\{\coqdocvar{Top}.95\}

\coqdocemptyline

\texttt{\coqcl~ Top.94}

~~~~~~~\texttt{Top.95 |= Top.94 < Top.95}

~~~~~~~~~~~~~~~~~~~~~~~~\texttt{Top.95 < Top.93 ~\coqcr}



The Coq's unifier algorithm in this case looks for a universe levels
\coqdocvar{Top}.95, which can be larger than the level of \coqdocdefinition{$R$}'s argument level
\coqdocvar{Top}.94, but smaller than the one of \coqdocdefinition{$R$} itself (i.e., \coqdocvar{Top}.93). In
the absence of other constraints, such system of equalities is easily
satisfiable.


However, the attempt to apply \coqdocdefinition{$R$} to \textit{itself} immediately leads to an
error reported, as the system cannot infer the level of the result, by
means of solving a system of universe level inequations, therefore,
preventing meta-circular paradoxes.






\coqdoceol
\coqdocemptyline
\coqdocnoindent
\coqdockw{Check} \coqdocdefinition{$R$} \coqdocdefinition{$R$}.\coqdoceol
\coqdocnoindent
\coqdoceol
\coqdocnoindent
\coqdocvar{The} \coqdocvar{term} "R" \coqdocvar{has} \coqdockw{type} "forall A : Type@\{Top.93\}, A -> A"\coqdoceol
\coqdocnoindent
\coqdocvar{while} \coqdocvar{it} \coqdockw{is} \coqdocvar{expected} \coqdocvar{to} \coqdocvar{have} \coqdockw{type} "?A"\coqdoceol
\coqdocnoindent
(\coqdocvar{unable} \coqdocvar{to} \coqdocvar{find} \coqdocvariable{a} \coqdocvar{well}-\coqdocvar{typed} \coqdocvar{instantiation} \coqdockw{for} "?A": \coqdocvar{cannot} \coqdocvar{ensure} \coqdocvar{that}\coqdoceol
\coqdocnoindent
"Type@\{Top.93+1\}" \coqdockw{is} \coqdocvariable{a} \coqdocvar{subtype} \coqdockw{\texttt{\emph{of}}} "Type@\{Top.93\}").

\coqdocemptyline


The solution to this problem is to think of the two occurrences \coqdocdefinition{$R$} in
the last example as of inhabitants of \textit{different} universes, so that
the \coqdocdefinition{$R$}-``function'' belongs to the universe with a higher level number
than \coqdocdefinition{$R$}-``argument''. Checking this scenario requires supporting a more
flexible form of universe polymorphism, which can assign different
universe levels to different occurrences of the same definition in a
common expression, and in this sense reminds \textit{let-polymorphism}
~\cite[\S 22.7]{Pierce:BOOK02}\index{let-polymorphism}.  This
feature was introduced in Coq since version
8.5~\cite{Sozeau-Tabareau:ITP14}, and it allows us to redefine \coqdocdefinition{$R$}
as universe-polymorphic via the new \coqdocvar{Polymorphic} \ccom{Polymorphic}
keyword.\footnote{More documentation on universe polymorphism is
available
at \url{https://coq.inria.fr/distrib/V8.5beta2/refman/Reference-Manual032.html}.}


\begin{coqdoccode}
\coqdocemptyline
\coqdocnoindent
\coqdocvar{Polymorphic} \coqdockw{Definition} \coqdocvar{RPoly} \{\coqdocvar{A} : \coqdockw{Type}\} (\coqdocvar{a} : \coqdocvar{A}) : \coqdocvar{A} := \coqdocvar{a}.\coqdoceol
\coqdocemptyline
\coqdocnoindent
\coqdockw{About} \coqdocvar{RPoly}.\coqdoceol
\coqdocemptyline
\end{coqdoccode}
\coqdoceol
\coqdocemptyline
\coqdocnoindent
\coqdocdefinition{RPoly} : \coqdockw{\ensuremath{\forall}} \coqdocvariable{A} : \coqdocvar{Type@}\{\coqdocvar{Top}.96\}, \coqdocvariable{A} \ensuremath{\rightarrow} \coqdocvariable{A}

\coqdocemptyline

\texttt{\coqcl~Top.96 |= ~\coqcr}



\coqdoceol
\coqdocemptyline
\coqdocnoindent
\coqdocdefinition{RPoly} \coqdockw{is} \coqdocvar{universe} \coqdocvar{polymorphic}\coqdoceol
\coqdocnoindent
\coqdocvar{Argument} \coqdocvariable{A} \coqdockw{is} \coqdocvar{implicit} \coqdocinductive{and} \coqdocvar{maximally} \coqdocvar{inserted}\coqdoceol
\coqdocnoindent
...

\coqdocemptyline


We can now apply \coqdocdefinition{RPoly} to itself using the following syntax.
\begin{coqdoccode}
\coqdocemptyline
\coqdocnoindent
\coqdockw{Definition} \coqdocvar{selfRPoly} := \coqdocvar{RPoly} (@\coqdocvar{RPoly}).\coqdoceol
\coqdocemptyline
\end{coqdoccode}
\coqdoceol
\coqdocemptyline
\coqdocnoindent
\coqdocdefinition{selfRPoly} \coqdockw{is} \coqdocvar{defined}

\coqdocemptyline


Let us now check the details of universes participating in
\coqdocdefinition{selfRPoly}'s typing.


\begin{coqdoccode}
\coqdocemptyline
\coqdocnoindent
\coqdockw{Print} \coqdocvar{selfRPoly}.\coqdoceol
\coqdocemptyline
\end{coqdoccode}
\coqdoceol
\coqdocemptyline
\coqdocnoindent
\coqdocdefinition{selfRPoly} = \coqdoceol
\coqdocnoindent
\coqdocvar{RPoly@}\{\coqdocvar{Top}.97\} (@\coqdocvar{RPoly@}\{\coqdocvar{Top}.98\})\coqdoceol
\coqdocindent{2.50em}
: \coqdockw{\ensuremath{\forall}} \coqdocvariable{A} : \coqdocvar{Type@}\{\coqdocvar{Top}.98\}, \coqdocvariable{A} \ensuremath{\rightarrow} \coqdocvariable{A}

\coqdocemptyline
 
\texttt{\coqcl ~ Top.97}

~~~~\texttt{   Top.98 |= Top.98 < Top.97}

~~~~~\texttt{~\coqcr}



The display above demonstrates that the two occurrences of \coqdocdefinition{RPoly}
were assumed by the typing algorithm to belong to different universes:
\coqdocvar{Top}.97 for the function and \coqdocvar{Top}.98 for the argument,
correspondingly. In the absence of additional additional constraints,
the inequality between them can be trivially satisfied by assuming
\coqdocvar{Top}.98 < \coqdocvar{Top}.97.


\begin{coqdoccode}
\coqdocemptyline
\coqdocemptyline
\coqdocemptyline
\end{coqdoccode}
